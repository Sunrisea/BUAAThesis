%% %%=================================================================
%% %% <UTF-8>
%% %% 北航学位论文模板使用样例
%% %% 请将以下文件与此LaTeX文件放在同一目录中.
%% %%-----------
%% %% buaa.cls                  : LaTeX宏模板文件
%% %% bst/GBT7714-2005.bst      : 国标参考文献BibTeX样式文件2005(https://github.com/Haixing-Hu/GBT7714-2005-BibTeX-Style)
%% %% bst/GBT7714-2015.bst      : 国标参考文献BibTeX样式文件2015(https://github.com/zepinglee/gbt7714-bibtex-style)
%% %% pic/logo-buaa.eps         : 论文封皮北航字样
%% %% pic/head-doctor.eps       : 论文封皮学术博士学位论文标题(华文行楷字体替代解决方案)
%% %% pic/head-prodoctor.eps    : 论文封皮专业博士学位论文标题(华文行楷字体替代解决方案)
%% %% pic/head-master.eps       : 论文封皮学术硕士学位论文标题(华文行楷字体替代解决方案)
%% %% pic/head-professional.eps : 论文封皮专业硕士学位论文标题(华文行楷字体替代解决方案)
%% %% tex/*.tex                 : 本模板样例中的独立章节
%% %%-----------
%% %% 请统一使用UTF-8编码.
%% %%=================================================================

%=================================================================
% buaa基于ctexbook模板
% 论文样式参考自《研究生手册--二〇二〇年七月》
%======================
% 模板导入:
% \documentclass[thesis,permission,printtype,ostype,<ctexbookoptions>]{buaa}
%======================
% 模板选项:
%======================
% I.论文类型(thesis)
%--------------------
% a.学术硕士论文(master)[缺省值]
% b.专业硕士论文(professional)
% c.学术博士论文(doctor)
% d.专业博士论文(prodoctor)
%--------------------
% II.密级(permission)
%--------------------
% a.公开(public)[缺省值]
% b.内部(privacy)
% c.秘密(secret=secret3)
% c.1.秘密3年(secret3)
% c.2.秘密5年(secret5)
% c.3.秘密10年(secret10)
% c.4.秘密永久(secret*)
% d.机密(classified=classified5)
% d.1.机密3年(classified3)
% d.2.机密5年(classified5)
% d.3.机密10年(classified10)
% d.4.机密永久(classified*)
% e.绝密(topsecret=topsecret10)
% e.1.绝密3年(topsecret3)
% e.2.绝密5年(topsecret5)
% e.3.绝密10年(topsecret10)
% e.4.绝密永久(topsecret*)
%--------------------
% III.打印设置(printtype)
%--------------------
% a.单面打印(oneside)[缺省值]
% b.双面打印(twoside)
%--------------------
% IV.系统类型(ostype)
%--------------------
% a.win(oneside)[缺省值]
% b.linux (linux)
% c.mac (mac)
%--------------------
% V.ctexbook设置选项(<ctexbookoptions>)
%--------------------
% ...
%======================
% 其他说明:
% 1. Mac系统请使用mac选项,并使用XeLaTeX编译。
% 2. 可加入额外ctexbook文档类的选项,其将会被传递给ctexbook。
%    例如:\documentclass[fontset=founder]{buaa}
% 3. CTeX在Linux下默认使用Fandol字体,为避免某些生僻字无法显示,在系统已安装方正
%    字体的前提下可通过fontset=founder选项常用方正字体。
%=================================================================

% !TeX program = xelatex

\documentclass[master,privacy,twoside,win]{buaa}

%=================================================================
% 开启/关闭引用编号颜色:参考文献,公式,图,表,算法 等……
\refcolor{off}   % 开启: on[默认]; 关闭: off;
% 摘要和正文从右侧页开始
\beginright{off} % 开启: on[默认]; 关闭: off;
% 空白页留字
\emptypagewords{[ -- This page is a preset empty page -- ]}

%=================================================================
% buaa模板已内嵌以下LaTeX工具包:
%--------------------
% ifthen, etoolbox, titletoc, remreset,
% geometry, fancyhdr, setspace,
% float, graphicx, subfigure, epstopdf,
% array, enumitem,
% booktabs, longtable, multirow, caption,
% listings, algorithm2e, amsmath, amsthm,
% hyperref, pifont, color, soul,
% ---
% For Win: times
% For Lin: newtxtext, newtxmath
% For Mac: times, fontspec
%--------------------
% 请在此处添加额外工具包>>


%=================================================================
% buaa模板已内嵌以下LaTeX宏:
%--------------------
% \highlight{text} % 黄色高亮
%--------------------
% 请在此处添加自定义宏>>


%%=================================================================
% 论文题目及副标题-{中文}{英文}
\Title{北航硕博士学位论文~\LaTeX{}模板\BUAAThesis{}}{\LaTeX{} Template of Beihang University Thesis \BUAAThesis{}}
\Subtitle{版本 \BUAAThesisVer{}}{Version \BUAAThesisVer{}}

% 学科大类,默认工学
% \Branch{工学}

% 院系,专业及研究方向
\Department{宇航学院}
\Major{控制科学与工程}
\Feild{模式识别与智能系统}

% 导师信息-{中文名}{英文名}{职称}
\Tutor{导师姓名}{Tutor}{教授}
\Cotutor{副导师姓名}{Cotutor}{高工}

% 学生姓名-{中文名}{英文名}
\Author{学生姓名}{Student}
% 学生学号
\StudentID{ID123456}

% 中图分类号
\CLC{TP391.4}

% 时间节点-{月}{日}{年}
\DateEnroll{09}{01}{2015}
\DateGraduate{03}{31}{2018}
\DateSubmit{01}{10}{2018}
\DateDefence{03}{01}{2018}

%%=================================================================
% 摘要-{中文}{英文}
\Abstract{%
  摘要是学位论文内容的简短陈述,应体现论文工作的核心思想。论文摘要应力求语言精炼准确。博士学位论文的中文摘要一般约800$\sim$1200字;硕士学位论文的中文摘要一般约500字。摘要内容应涉及本项科研工作的目的和意义、研究思想和方法、研究成果和结论。博士学位论文必须突出论文的创造性成果,硕士学位论文必须突出论文的新见解。

  关键字是为用户查找文献,从文中选取出来揭示全文主体内容的一组词语或术语,应尽量采用词表中的规范词(参考相应的技术术语标准)。关键词一般3$\sim$5个,按词条的外延层次排列(外延大的排在前面)。关键词之间用逗号分开,最后一个关键词后不打标点符号。

  为了国际交流的需要,论文必须有英文摘要。英文摘要的内容及关键词应与中文摘要及关键词一致,要符合英语语法,语句通顺,文字流畅。英文和汉语拼音一律为Times New Roman体,字号与中文摘要相同。
  }{%
  What were you doing 500 years ago? Oh, that's right nothing, because you didn't exist yet. In fact, several generations of your family had yet to leave their mark on the world, but one very special shark may already have been swimming in the chilly North Atlantic at that time, and the incredible animal is somehow still alive today.

  Scientists studying Greenland sharks observed the particularly old specimen just recently, and after studying it they've determined that the creature is approximately 272 to 512 years old. That's an absolutely insane figure, and if its age lands towards the higher end, it makes the animal the oldest observed living vertebrate on the entire planet.

  Greenland sharks are an incredible species in a number of ways, but most notable is its longevity. The sharks are well over 100 years old before even reaching sexual maturity, and regularly live for centuries. This particularly old specimen, along with 27 others, were analyzed using radiocarbon dating. The reading came back at around 392 years, but potential margin of error means the animal's true age is somewhere between 272 and 512.

  The shark, which is a female, measures an impressive 18 feet long. That's pretty large, but it might not sound particularly large for an ocean-dwelling creature that lives hundreds of years. That is, until you consider that the Greenland shark only grows around one centimeter per year. With that in mind, 18 feet is actually downright massive.

  As for how this particular shark species manages to live so incredibly long, scientists attribute a lot of its longevity to its sluggish metabolism, as well as its environment. The frigid waters where the sharks thrive is thought to increase overall lifespan in a variety of ways. Past research has shown that cold environments can help slow aging, and these centuries-old sharks are most certainly benefiting from their chilly surroundings.

  --- Online news {\it Scientists find incredible shark that may be over 500 years old and still kicking\/}, 12.16.2017. (http://bgr.com/2017/12/14/oldest-shark-greenland-512-years-old/).
}
% 关键字-{中文}{英文}
\Keyword{%
    北航,学位论文,博士,硕士,中文,\LaTeX{}模板,\BUAAThesis{}
  }{%
    News, BGR, Shark
}

% 图标目录
\Listfigtab{on} % 启用: on[默认]; 双标题:bi; 关闭: off;

% 缩写定义 按tabular环境或其他列表环境编写
% \Abbreviations{ \centering
% \begin{tabular}{cl}
%   $E$ & 能量 \\
%   $m$ & 质量 \\
%   $c$ & 光速 \\
%   $P$ & 概率 \\
%   $T$ & 时间 \\
%   $v$ & 速度 \\
% \end{tabular}
% }

\begin{document}

%%=================================================================
% 标题级别
%--------------------
% \chapter{第一章}
% \section{1.1 小节}
% \subsection{1.1.1 条}
% \subsubsection{1.1.1.1}
% \paragraph{1.1.1.1.1}
% \subparagraph{1.1.1.1.1.1}
%--------------------
%%=================================================================

% 绪论
\input{tex/chap_intro}

% 说明
\input{tex/chap_instruction}

% 示例
\input{tex/chap_sample}

% 总结
% !TeX root = ../Template.tex
% 总结
\summary
% 自从2017年被提出,Transformer已经在计算机视觉、自然语言处理等多个领域大放异彩,展现出了巨大的潜力,收到了许多的关注,因此也有不少工作尝试将Transformer网络引入到知识图谱补全领域中。
本课题针对传统知识图谱嵌入方法和基于图神经网络的方法的缺点,研究如何利用Transformer模型来学习知识图谱中的语义和结构信息,提升知识图谱补全任务的性能,提出了两种新型的基于Transformer的知识图谱补全方法:基于邻域感知的Transformer模型NATLP,并在NATLP模型的基础上进一步提出了结合图路径和局部邻域的Transformer模型TKGE-PN。

针对基于图神经网络的知识图谱嵌入方法表达能力不足的问题,NATLP研究利用Transformer网络来完成知识图谱补全。针对以往模型对关系和实体间的交互建模不足的问题,NATLP在模型输入信息构造阶段,基于关系生成特定的网络参数,实现关系特定的邻居信息构造,显示建模了不同关系对于实体传递消息的影响;针对Transformer无法直接感知图结构的问题,NALTP对Transformer的自注意力机制进行了改造,提出了一种融合图结构的自注意力机制,使得模型能够学习到输入消息之间的互相依赖;

NATLP模型解决了基于图神经网络的模型存在的部分问题,但依然无法有效学习图谱中的长距离依赖。针对基于图神经网络的方法和NATLP方法中的以上缺陷,本文在NATLP的基础上进一步提出了结合图路径和局部邻域的Transformer模型TKGE-PN。TKGE-PN通过对知识图谱中的局部邻域和图路径两种结构信息的融合,完成了在利用丰富的邻域信息的同时对于图谱中长距离信息的挖掘,提高了知识图谱补全任务的性能。TKGE-PN首先通过基于有偏随机游走的采样算法对图路径进行采样,随后通过基于Transformer的图路径编码模块学习其中的长距离依赖,并通过掩蔽实体关系任务实现长短距离信息的平衡;最后通过局部邻域编码模块实现了图路径和局部邻域结构信息的综合应用。

实验结果表明,在两个标准数据集WN18RR和FB15k-237上,NATLP和TKGE-PN的链路预测任务的性能表现超越绝大多数现有的嵌入模型,证明了本文提出的两个模型以及其中的关键设计的有效性。

% 未来,我们计划进一步探索对于知识图谱中结构信息的利用,包括更有效率的利用方式和更加丰富的结构信息种类,同时尝试将TKGE-PN应用到除链路预测之外的其他知识图谱表示学习任务中。

本文涉及的研究内容还有许多可以拓展的地方。首先,知识图谱的图结构除了图路径和局部邻域之外还有其他的表现形式,我们计划进一步探索对于知识图谱中结构信息的利用,例如在规模更大的子图范围上来应用Transformer网络,进一步挖掘模型的潜力。

其次,除了结构信息之外,知识图谱中也蕴含着丰富的文本信息,例如实体的文字描述。而Transformer网络被广泛应用于预训练语言模型中,在挖掘文本语义信息方面存在天然优势。未来我们计划研究将模型与预训练语言模型进行结合,同时利用文本信息和图结构信息来完成知识图谱补全任务。

最后,Transformer网络的复杂度较高,训练和预测时的开销较大,在处理大规模知识图谱时可能会遇到资源上的瓶颈。因此如何优化基于Transformer的结构带来的大量资源开销也是一个值得研究的方向。



% 参考文献
% 手册中参考文献标准似乎并没有严格按照国标GBT7714-2015执行
\Bib{bst/GBT7714-BUAA}{ref}

% 附录
\input{tex/chap_appendix}

% 攻读学位期间成果
% !TeX root = ../Template.tex
% [攻读学位期间取得的成果]
\achievement
\noindent
[1] Liu X, \textbf{Zhu T}, Tan H, et al. Heterogeneous graph neural network with hypernetworks
for knowledge graph embedding [C]//The Semantic Web–ISWC 2022: 21st International
Semantic Web Conference, Virtual Event, October 23–27, 2022, Proceedings. Berlin,
Heidelberg: Springer-Verlag, 2022: 284–302

\noindent
[2] 郭子溢,\textbf{朱桐},林广艳,等. 球面坐标下基于语义分层的知识图谱补全方法 [J]. 应用科学学报, 2024, 42 (01): 119-133.



% 致谢
% !TeX root = ../Template.tex
% [致谢]
\acknowledgments

研究生三年生涯转瞬即逝,我的学生生涯也马上要走到了尽头。回首过去的三年,我初次叩开了科研的大门,虽然未能深入探索,但也收获良多,受益匪浅。

在三年的研究生生活中,我首先要感谢的是谭火彬老师和林广艳老师对我的学术和生活上给予的指导和帮助。两位老师不仅会在学术上为我提供建议,指明方向,对我的论文提出了许多宝贵的修改意见,生活中,还时刻关心着我们的身体健康和心理状况,在我的职业规划上也提供了大力的支持。实验室的良好氛围也为我提供了非常好的学习和研究的环境。能够接受两位老师的指导是我的幸运。还要感谢我的父母,他们的默默付出是我最坚强的后盾,父母的理解和爱护是我低谷时最重要的慰藉和支持。

此外,我还要感谢实验室的各位同门。感谢刘希阳学长在我初入实验室时提供的指导,帮助我迈出了学术研究的第一步,在我论文工作遇到困难时也对我提出了非常宝贵的意见和建议。还要感谢柳啸峰、郭子溢、朱伯同三位学长在我的学术研究中为我答疑解惑,学长们丰富的经验帮助我少走了不少弯路。感谢任博林和蒋沛宇两位学弟,论文实验的顺利完成离不开他们对于实验室服务器的维护。

最后,我要感谢我的小伙伴们,三年的时间我们一起度过了一段难忘的时光。




% 作者简介
\input{tex/chap_biography}

\vspace{5cm}

This is \BUAAThesis{}, Happy TeXing! --- from WeiQM.

\end{document}
