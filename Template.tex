%% %%=================================================================
%% %% <UTF-8>
%% %% 北航学位论文模板使用样例
%% %% 请将以下文件与此LaTeX文件放在同一目录中.
%% %%-----------
%% %% buaa.cls                  : LaTeX宏模板文件
%% %% bst/GBT7714-2005.bst      : 国标参考文献BibTeX样式文件2005(https://github.com/Haixing-Hu/GBT7714-2005-BibTeX-Style)
%% %% bst/GBT7714-2015.bst      : 国标参考文献BibTeX样式文件2015(https://github.com/zepinglee/gbt7714-bibtex-style)
%% %% pic/logo-buaa.eps         : 论文封皮北航字样
%% %% pic/head-doctor.eps       : 论文封皮学术博士学位论文标题(华文行楷字体替代解决方案)
%% %% pic/head-prodoctor.eps    : 论文封皮专业博士学位论文标题(华文行楷字体替代解决方案)
%% %% pic/head-master.eps       : 论文封皮学术硕士学位论文标题(华文行楷字体替代解决方案)
%% %% pic/head-professional.eps : 论文封皮专业硕士学位论文标题(华文行楷字体替代解决方案)
%% %% tex/*.tex                 : 本模板样例中的独立章节
%% %%-----------
%% %% 请统一使用UTF-8编码.
%% %%=================================================================

%=================================================================
% buaa基于ctexbook模板
% 论文样式参考自《研究生手册--二〇二〇年七月》
%======================
% 模板导入:
% \documentclass[thesis,permission,printtype,ostype,<ctexbookoptions>]{buaa}
%======================
% 模板选项:
%======================
% I.论文类型(thesis)
%--------------------
% a.学术硕士论文(master)[缺省值]
% b.专业硕士论文(professional)
% c.学术博士论文(doctor)
% d.专业博士论文(prodoctor)
%--------------------
% II.密级(permission)
%--------------------
% a.公开(public)[缺省值]
% b.内部(privacy)
% c.秘密(secret=secret3)
% c.1.秘密3年(secret3)
% c.2.秘密5年(secret5)
% c.3.秘密10年(secret10)
% c.4.秘密永久(secret*)
% d.机密(classified=classified5)
% d.1.机密3年(classified3)
% d.2.机密5年(classified5)
% d.3.机密10年(classified10)
% d.4.机密永久(classified*)
% e.绝密(topsecret=topsecret10)
% e.1.绝密3年(topsecret3)
% e.2.绝密5年(topsecret5)
% e.3.绝密10年(topsecret10)
% e.4.绝密永久(topsecret*)
%--------------------
% III.打印设置(printtype)
%--------------------
% a.单面打印(oneside)[缺省值]
% b.双面打印(twoside)
%--------------------
% IV.系统类型(ostype)
%--------------------
% a.win(oneside)[缺省值]
% b.linux (linux)
% c.mac (mac)
%--------------------
% V.ctexbook设置选项(<ctexbookoptions>)
%--------------------
% ...
%======================
% 其他说明:
% 1. Mac系统请使用mac选项,并使用XeLaTeX编译。
% 2. 可加入额外ctexbook文档类的选项,其将会被传递给ctexbook。
%    例如:\documentclass[fontset=founder]{buaa}
% 3. CTeX在Linux下默认使用Fandol字体,为避免某些生僻字无法显示,在系统已安装方正
%    字体的前提下可通过fontset=founder选项常用方正字体。
%=================================================================

% !TeX program = xelatex

\documentclass[master,oneside,max]{buaa}

%=================================================================
% 开启/关闭引用编号颜色:参考文献,公式,图,表,算法 等……
\refcolor{off}   % 开启: on[默认]; 关闭: off;
% 摘要和正文从右侧页开始
\beginright{off} % 开启: on[默认]; 关闭: off;
% 空白页留字
% \emptypagewords{[ -- This page is a preset empty page -- ]}

%=================================================================
% buaa模板已内嵌以下LaTeX工具包:
%--------------------
% ifthen, etoolbox, titletoc, remreset,
% geometry, fancyhdr, setspace,
% float, graphicx, subfigure, epstopdf,
% array, enumitem,
% booktabs, longtable, multirow, caption,
% listings, algorithm2e, amsmath, amsthm,
% hyperref, pifont, color, soul,
% ---
% For Win: times
% For Lin: newtxtext, newtxmath
% For Mac: times, fontspec
%--------------------
% 请在此处添加额外工具包>>


%=================================================================
% buaa模板已内嵌以下LaTeX宏:
%--------------------
% \highlight{text} % 黄色高亮
%--------------------
% 请在此处添加自定义宏>>


%%=================================================================
% 论文题目及副标题-{中文}{英文}
\Title{基于Transformer的知识图谱补全算法研究}{Research of Knowledge Graph Completion Algorithm Based on Transformer}
% \Subtitle{版本 \BUAAThesisVer{}}{Version \BUAAThesisVer{}}

% 学科大类,默认工学
% \Branch{工学}

% 院系,专业及研究方向
\Department{软件学院}
\Major{软件工程}
\Feild{软件工程}

% 导师信息-{中文名}{英文名}{职称}
\Tutor{谭火彬}{Tan Huobin}{副教授}
% \Cotutor{副导师姓名}{Cotutor}{高工}

% 学生姓名-{中文名}{英文名}
\Author{朱桐}{Zhu Tong}
% 学生学号
\StudentID{SY2121127}

% 中图分类号
\CLC{TP391.4}

% 时间节点-{月}{日}{年}
\DateEnroll{09}{01}{2021}
\DateGraduate{05}{16}{2024}
\DateSubmit{05}{10}{2024}
\DateDefence{05}{20}{2024}

%%=================================================================
% 摘要-{中文}{英文}
\Abstract{%
知识图谱在多个人工智能领域中得到了广泛的应用。但是由于现实世界不断发展带来的知识的动态变化,即使是大规模知识图谱也难以囊括所有的知识,影响了知识图谱在下游任务中的应用,因此知识图谱补全任务成为了热门的研究领域。知识图谱补全任务旨在自动化预测知识图谱中缺失的三元组,目前的主流方案是知识图谱嵌入,将实体和关系投影到低维向量空间中以学习知识图谱中的隐含规律。许多现有的方法通过利用图谱的图结构信息获得优异的性能,其中最具代表性的为利用图谱中局部邻域结构的基于图神经网络的方法。但是一方面图神经网络其浅层的网络结构限制了模型的表达能力,另一方面存在的过度平滑问题也导致模型无法利用长距离的信息。Transformer是注意力机制方面里程碑式的工作,基于Transformer的模型变体在计算机视觉和编程语言领域中表现出了出色的性能。因此针对上述不足,本文研究基于Transformer的知识图谱补全方法,首先提出了基于领域感知的Transformer模型(Neighborhood Aware Transformer for Link Prediction, NATLP),提高了模型的表达能力获得了更好的知识图谱补全性能,并在此基础上进一步提出了结合图路径和局部邻域的Transformer模型 (A Transformer-based Knowledge Graph Embedding Model Combining
Graph Paths and Local Neighborhood,TKGE-PN),实现了对于知识图谱中长短距离依赖的综合学习。本文的主要工作内容如下:

(1)针对基于图神经网络的知识图谱嵌入方法表达能力不足、无法捕捉邻居实体之间的相互依赖的问题,本文提出了基于领域感知的Transformer模型NATLP,通过关系特定的邻居实体信息构造建模了不同关系对于实体传递消息的影响,并对Transformer的自注意力计算机制进行了改造,使Transformer能够感知到局部邻域内的图结构数据,学习到邻居实体之间的相互依赖。

(2)针对基于图神经网络的方法以及NATLP无法学习到长距离依赖的问题,本文提出了结合图路径和局部领域的Transformer模型TKGE-PN。模型首先通过有偏随机游走算法以中心实体为起点采样多条图路径,其次通过基于Transformer的图路径编码模块学习图路径中的长距离依赖并转化为向量表示,最后局部邻域编码模块集合所有图路径向量结合局部邻域信息表示对三元组打分。在图路径编码的过程中,利用掩蔽实体关系预测任务增强模型对于长距离信息的学习能力。

论文在两个主流基准数据集FB15K-237和WN18RR上对提出的NATLP和TKGE-PN模型进行了实验,实验结果表明本文提出的两个模型在两个数据集上的多项指标超越了绝大部分基线模型,证明了本文提出的关键设计能够有效提高知识图谱补全任务的性能。
  }{Knowledge graphs have been extensively applied across multiple artificial intelligence domains. However, dynamic changes in knowledge due to the continuous development of the real world make it challenging for even large-scale knowledge graphs to encompass all information, affecting their application in downstream tasks. Consequently, the task of knowledge graph completion has emerged as a popular research area. Knowledge graph completion aims at automatically predicting missing triples within the knowledge graphs, and the mainstream solution currently is knowledge graph embedding, which projects entities and relations into a low-dimensional vector space to learn the implicit patterns within the knowledge graph. Many existing methods have achieved remarkable performance by utilizing the graph structure \\    information of the graph, with the most representative being methods based on graph neural networks that utilize local neighborhood structures within the graph. Nevertheless, the shallow network structure of graph neural networks limits the model's expressive power, and the existing over-smoothing problem also prevents the model from utilizing long-range information. Transformer represents a milestone in attention mechanisms, and Transformer-based model variants have demonstrated exceptional performance in the fields of computer vision and programming languages. In response to the aforementioned shortcomings, this paper investigates Transformer-based knowledge graph completion methods. We initially propose the Neighborhood Aware Transformer for Link Prediction (NATLP), which improves the model's expressive power to achieve better knowledge graph completion performance. Building on this, we further propose a Transformer-based Knowledge Graph Embedding Model Combining Graph Paths and Local Neighborhood (TKGE-PN), achieving comprehensive learning of both short and long-distance dependencies within the knowledge graph. The main contributions of this paper are as follows:

  (1) Addressing the issue of insufficient expressive capabilities of knowledge graph embedding methods based on graph neural networks and their inability to capture the interdependencies between neighbor entities, this paper proposes the NATLP. By constructing and modeling the influence of different relations on entity message passing using relationship-specific neighbor entity information, we modify the Transformer's self-attention mechanism to enable the Transformer to perceive the graph structure data within the local neighborhood, thereby learning the \\ interdependencies between neighbor entities.
  
  (2) To address the problem that graph neural network-based methods and NATLP cannot learn long-distance dependencies, this paper proposes TKGE-PN, a Transformer-based model combining graph paths and local neighborhood. The model first samples multiple graph paths starting from the central entity using a biased random walk algorithm. Next, it learns the long-distance dependencies within graph paths through a Transformer-based graph path encoding module, converting them into vector representations. Finally, a local neighborhood encoding module aggregates all graph path vectors with local neighborhood information to score triples. During graph path encoding, a masked entity-relation prediction task is utilized to enhance the model's ability to learn long-distance information.
  
  Experiments on two mainstream benchmark datasets, FB15K-237 and WN18RR, are conducted to evaluate the proposed NATLP and TKGE-PN models. The experimental results show that both models proposed in this paper surpass the majority of baseline models on several metrics on both datasets, proving that the key designs proposed in this paper can effectively improve the performance of knowledge graph completion tasks.



}
% 关键字-{中文}{英文}
\Keyword{%
    知识图谱,知识图谱补全,知识图谱嵌入,Transformer,图路径,局部邻域
  }{%
    Knowledge Graph, Knowledge Graph Completion,
Knowledge Graph Embedding, Transformer, Graph Path, Local Neighborhood

}
% Given the widespread application of Knowledge Graphs (KGs) across multiple Artificial Intelligence domains, the incompleteness of most current knowledge graphs has made Knowledge Graph Completion (KGC) a hot research area. KGC aims to predict missing triples in a knowledge graph, and the current mainstream solution is Knowledge Graph Embeddings (KGE), which projects entities and relations into a low-dimensional vector space to learn the hidden patterns within the knowledge graph. Many existing methods achieve excellent performance by utilizing the graph structure information of the graph, among which the most representative is the method based on Graph Neural Networks (GNNs) that utilizes the local neighborhood structure in the graph. However, the shallow network structure of GNNs limits the model's expressive power on one hand, and on the other hand, the issue of over-smoothing also prevents the model from utilizing long-distance information. Transformer, a milestone work in the field of attention mechanisms, and its variants have shown outstanding performance in the fields of Computer Vision and Programming Languages. Therefore, to address the aforementioned deficiencies, this paper investigates Transformer-based methods for Knowledge Graph Completion, introducing a domain-aware Transformer model, NATLP, and a Transformer model that combines graph paths and local neighborhoods, TKGE-PN. The main contributions of this paper are as follows:

% (1) To address the issue of limited expressive power and inability to capture mutual dependencies among neighboring entities in knowledge graph embedding methods based on GNNs, this paper proposes NATLP, a neighborhood aware transformer for Link Prediction that constructs and models the impact of different relations on entity message passing based on relation-specific neighboring entity information. The Transformer's self-attention mechanism is also modified to enable it to perceive local neighborhood graph structure data and learn mutual dependencies among neighboring entities.

% (2) To address the issue that methods based on Graph Neural Networks, as well as NATLP, fail to learn long-distance dependencies, this paper introduces a transformer-based knowledge graph embedding Model combining Graph Paths and Local Neighborhood, TKGE-PN. The model first samples multiple graph paths centered on the target entity using a biased random walk algorithm. Then, it employs a Transformer-based graph path encoding module to learn and vectorize the long-distance dependencies within these paths. Lastly, a local neighborhood encoding module aggregates all the graph path vectors along with local neighborhood information to score triples. During the graph path encoding process, the model's ability to learn long-distance information is enhanced by employing a masked entity relation prediction task.

% (3) Experiments on two mainstream benchmark datasets, FB15K-237 and WN18RR, demonstrated the effectiveness of the proposed NATLP and TKGE-PN models along with the key designs.



% 图标目录
\Listfigtab{on} % 启用: on[默认]; 双标题:bi; 关闭: off;

% 缩写定义 按tabular环境或其他列表环境编写
% \Abbreviations{ \centering
% \begin{tabular}{cl}
%   $E$ & 能量 \\
%   $m$ & 质量 \\
%   $c$ & 光速 \\
%   $P$ & 概率 \\
%   $T$ & 时间 \\
%   $v$ & 速度 \\
% \end{tabular}
% }

\begin{document}

%%=================================================================
% 标题级别
%--------------------
% \chapter{第一章}
% \section{1.1 小节}
% \subsection{1.1.1 条}
% \subsubsection{1.1.1.1}
% \paragraph{1.1.1.1.1}
% \subparagraph{1.1.1.1.1.1}
%--------------------
%%=================================================================

% 绪论
\chapter{绪论}

\section{研究背景与意义}
论文选题来源于实验室承担的国家重点研发计划等多个课题,需要面向材料、建筑等不同领域场景建立行业知识图谱。本文主要研究高效的知识图谱补全算法,为课题中各类行业知识图谱的构建和下游任务的应用提供理论和技术支撑。

知识图谱(Knowledge Graph, KG)是知识库的一种主要形式,是由事实三元组(头实体、关系、尾实体)表示的结构化知识的集合,在多个人工智能领域中得到了广泛的应用,例如语义搜索\upcite{yuyisousuo}、问答\upcite{wenda}和推荐系统\upcite{recommendation_system}。主流的开放知识图谱包括FreeBase\upcite{freebase},Wikidata\upcite{Wikidata},DBpedia\upcite{DBpedia}和YAGO\upcite{YAGO}等,它们通常包含由数十亿个实体和关系所构建的海量事实。然而,由于现实世界不断发展带来的知识的动态变化,即使是大规模知识图谱也难以囊括所有的知识,这限制了知识图谱在现实世界中的应用。因此,近年知识图谱补全(Knowledge Graph Completion, KGC),又称链路预测任务,成为了知识图谱领域的热门研究方向,尝试在给定事实三元组中的头(尾)实体和关系的情况下,自动预测缺失的尾(头)实体。

% 论文选题来源于国家重点研发计划课题“稀土催化材料专用数据库及全流程数字化研发平台”,本文研究高效的知识图谱补全方法,为稀土催化材料知识图谱构建和应用提供技术支撑。

% 稀土是由镧系元素和与其密切相关的钪和钇等化学元素组成,稀土的存在能有效提高催化剂的储氧能力、提高活性金属的分散度、降低贵金属用量、促进水气转化和蒸汽重整反应等性质,在催化材料领域中有着重要的应用。
% % 知识图谱(Knowledge Graph, KG)是知识库的一种主要表现形式,是由事实三元组(头实体、关系、尾实体)表示的结构化知识的集合,在多个领域中已经得到了广泛的应用。
% 稀土催化材料知识图谱存储着稀土催化材料的合成方式、理化性质、组成结构等重要信息,被应用于稀土催化材料合成等任务中,能够帮助稀土催化材料降低开发成本、减少开发周期。然而,由于稀土材料领域的不断研究发展带来的知识的动态变化,和大多数知识图谱一样,稀土催化材料知识图谱往往是不完全的,难以囊括领域内的所有知识,对其在上层任务中的应用造成了阻碍。因此,为了挖掘图谱内蕴含的丰富信息,完善稀土催化材料知识图谱,知识图谱补全(Knowledge Graph Completion, KGC),又称链路预测任务,成为了知识图谱领域的热门研究方向。

知识图谱补全需要挖掘图谱中隐藏的语义信息,但是知识图谱中的事实三元组一般是以文本形式进行储存的,无法直接利用,需要首先寻找一种合适的方式来对语义信息进行表达。传统方法一般通过特征工程进行,效率低且可移植性较差,因此不少研究者投入了自动化知识补全的研究。

目前,知识图谱补全的主流解决方案是知识图谱嵌入(Knowledge Graph Embedding, KGE),又称知识图谱表示学习\upcite{TransE}(Knowledge Graph Representation Learning)。它的核心思想是将图谱中的实体和关系投影到低维向量空间中,通过预先设计好的得分函数(Scoring Function)评估事实三元组的合理性,并基于知识图谱中的已有事实,最大化对正确事实三元组的预测概率。通过这种方式获得的嵌入表示不仅可以用于知识图谱补全,还能够用于语义搜索、问答和推荐系统等下游任务中。

传统的知识图谱表示学习方法主要考虑如何在单纯的三元组上进行学习,但这种方式存在较大的缺陷:
% 知识图谱本身是一个带标签的有向图,单纯考虑三元组的方法忽略了三元组之间的联系,
忽略了知识图谱本身的图结构信息。基于图神经网络的模型通过学习中心实体的局部邻域结构一定程度上解决了以上的问题,获得了更加优秀的性能,但依然存在不足:首先图神经网络的网络结构较浅,限制了模型的表达能力;另外,
% 基于图神经网络的方法的感受域与其网络层数相关,随着图神经网络层数的增加,相邻节点的感受域重叠程度升高,现有的方法会遭遇过度平滑的问题而导致性能下降,
基于图神经网络的方法随着网络层数的提升会遭遇过度平滑\upcite{over-smoothing}的问题,导致其只能捕捉单个实体附近1-2跳内的局部邻域信息,缺乏利用长距离依赖的能力。针对以上问题,本文研究基于Transformer的知识图谱补全方法。Transformer\upcite{Transformer}被公认为是建模序列数据的最强大的神经网络,不少工作致力于研究将Transformer网络应用到知识图谱嵌入工作中。本文研究利用Transformer强大的表达能力,结合知识图谱中图结构信息,实现更加准确的知识图谱补全,支持各类行业知识图谱的构建和应用。


\section{国内外研究现状}
知识图谱(Knowledge Graph, KG)的现代含义由2012年谷歌知识图谱\upcite{googlekg}的发布而确立,它是知识库的一种主要表现形式,是由事实三元组(头实体、关系、尾实体)表示的结构化知识的集合。图谱中的节点为实体,表示现实世界中的具体事物;图谱中的节点为关系,表示实体之间的联系。目前知识图谱已经在多个人工智能领域中得到了广泛的应用,例如语义搜索\upcite{yuyisousuo}、问答\upcite{wenda}和推荐系统\upcite{recommendation_system}。主流的开放知识图谱包括FreeBase\upcite{freebase},Wikidata\upcite{Wikidata},DBpedia\upcite{DBpedia}和YAGO\upcite{YAGO}等,它们通常包含使用数十亿个实体和关系构建的大量事实。然而由于现实世界不断发展,随时都有新的知识产生和旧的知识失效,即使是大规模知识图谱也不可避免的缺乏部分事实,这限制了知识图谱在现实世界中的应用。因此,近年知识图谱补全又称链路预测任务,成为了知识图谱领域的热门研究方向,尝试在给定事实三元组中的头(尾)实体和关系的情况下,自动预测缺失的尾(头)实体。

目前,知识图谱嵌入是知识图谱补全任务的主流解决方案,它将图谱中的实体和关系转化为低维向量空间中的向量,尽可能地保留其原始的结构性质,并用得分函数估计事实三元组正确的概率。现阶段对于知识图谱嵌入算法的研究,根据方法的核心思想和实现方式的不同,可以划分为传统的知识图谱嵌入方法,基于图神经网络的知识图谱嵌入方法,基于Transformer的知识图谱嵌入方法和融合多源信息的知识图谱嵌入方法。
\subsection{传统的知识图谱嵌入方法}
传统的知识图谱嵌入方法仅独立研究知识图谱中的事实三元组,主要包含基于翻译的方法、基于张量分解的方法和引入神经网络后的基于多层神经网络方法、基于卷积神经网络的方法。

基于翻译的方法是最早被提出的一类知识图谱嵌入方法,最早起源于2013年的TransE\upcite{TransE}模型,核心思想是将知识图谱中的关系视为一个实体到另一个实体的翻译,又被称为平移距离模型。由于TransE无法有效建模知识图谱中的一对多、多对一、多对多关系,后续基于TransE进行改进并衍生出了如TransH\upcite{TransH}、TransR\upcite{TransR}、TransD\upcite{TransD}等模型,不断丰富模型的表达能力。基于翻译的方法最大的优点在于其模型结构简单、计算速度快、易于理解且可解释性较强,但浅层的模型结构也限制了模型的表达能力。

以RESCAL\upcite{RESCAL}为代表的基于张量分解的方法则将整个知识图谱表示为一个高维的稀疏张量,通过对其进行张量分解来获得实体和关系的嵌入。RESCAL用一个维度为$N\times N\times M$的张量来表示一个实体数量为$N$,关系数量为$M$的知识图谱,通过对其进行张量分解,模型最终能够得到用一维向量表示的实体嵌入和用二维矩阵表示的关系嵌入。继RESCAL之后,基于张量分解的思想提出的DistMult\upcite{DistMult}、ComplEx\upcite{ComplEx}、ANALOGY\upcite{ANALOGY}等一系列模型分别从强化模型表达能力以及压缩模型参数两方面对RESCAL模型进行了改进。
% DistMult将关系嵌入进行了简化,选用了对角矩阵替代了RESCAL的二维矩阵,降低了模型的复杂度并获得了更优的性能;ComplEx则将模型从实数域扩展到了复数域,提高了模型的表达能力。
总的来说,基于张量分解的方法可解释性较强,并能够捕捉到实体和关系之间的双线性关系,但和基于翻译的方法类似,浅层的模型结构很难有效的学习图谱中蕴含的复杂信息,模型表达能力较弱。

而随着神经网络的发展,大量基于神经网络的知识图谱嵌入方法开始涌现。使用神经网络进行知识图谱嵌入能够建立更加复杂的模型,模型的表达能力更强,并且可以自动学习知识图谱当中蕴含的特征。这其中最早被提出的是基于多层神经网络的方法,SME\upcite{SME}、NTN\upcite{NTN}、MLP\upcite{MLP}等模型直接使用多层的神经网络去拟合知识图谱。
% 以事实三元组的嵌入作为模型的输入,输出三元组正确的概率。
这类方法相较之前没有神经网络结构的方法在性能上有了提升,但网络结构相对简单,可解释性较差。



而随后受到计算机视觉领域的研究方法的启发,有不少工作尝试引入卷积操作,大量基于卷积神经网络的方法被提出,其中最具代表性的方法为ConvE\upcite{ConvE}。ConvE将事实三元组中的头实体和关系的一维向量嵌入,重组为二维张量并对其进行卷积操作,将结果向量化之后经过神经网络层,随后和候选实体的嵌入进行点乘,输出事实三元组的正确概率。基于ConvE的思想,有不少方法提出了进一步的改进。ConvR\upcite{ConvR}使用关系嵌入构造卷积核,减少了网络的参数;ConvKB\upcite{ConvKB}通过在实体和关系的相同维度上进行卷积,能够捕获在实体和关系之间相同维度上的联系;InteractE\upcite{InteractE}则优化了二维张量的塑造方式,大大提升了头实体和关系之间的交互。

\subsection{基于图神经网络的知识图谱嵌入方法}

以上提到的知识图谱嵌入方法研究的对象是知识图谱中独立的三元组,这导致这些模型忽略了知识图谱的结构信息,因此被统一归类为传统的知识图谱嵌入方法。例如,这些方法没有办法感知到头实体的邻居实体,无法充分利用每个实体丰富的邻域结构,不仅链路预测的性能受到限制,而且也缺乏嵌入空间的可解释性。而基于图神经网络(Graph Neural Network, GNN)的知识图谱嵌入方法则利用图卷积神经网络来捕获图谱中的图结构信息,中心实体接受来自邻居实体与邻居关系的消息,并依此对实体和关系的嵌入表示进行更新。

R-GCN\upcite{R-GCN}是第一个利用图卷积神经网络学习知识图谱表示的方法,整体采用编码器-解码器架构。编码器部分通过图神经网络对图结构进行建模,在R-GCN的信息传播过程中,中心实体会接受来自出边、入边和自循环边三个方向的信息;通过多次信息传播模型能够获得多阶邻居的信息。解码器部分则基于编码的信息对三元组进行打分。后续提出的基于图神经网络的方法沿用了R-GCN的编码器-解码器架构,并在此基础上进行改进。SACN\upcite{SACN}模型基于关系类型将实体的邻域划分为带权值的子图进行聚合。TransGCN\upcite{TransGCN}提出了两种基于翻译的思想的编码器同时学习实体和关系嵌入,分别用于实数域和复数域。

而受到自然语言处理和计算机视觉领域中注意力机制的成功的启发,有不少工作尝试将注意力机制引入到基于图神经网络的知识图谱嵌入方法中来并取得了不错的效果。KBGAT\upcite{KBGAT}是首个在知识图谱嵌入领域应用图注意力网络的方法,模型能够自动分辨出哪些邻居实体的信息对于中心实体是更加重要的。RGHAT\upcite{RGHAT}将注意力机制进行了进一步的细分,引入了关系注意力机制和实体注意力机制,实现了更细粒度的建模。EIGAT\upcite{EIGAT}则通过随机游走算法引入了全局实体重要性,将局部注意力机制和知识图谱的全局信息进行了结合。

% 基于图神经网络的方法的成功证明了实体的局部邻域蕴含了丰富的信息,但知识图谱的结构信息除了常见的局部邻域之外还有多种表达形式,例如图路径和子图等。在知识图谱中,图路径被定义为图谱中的实体-关系链,例如(Yao Ming, Born In, Shanghai, City Of, China)。和基于图神经网络的方法利用中心实体的局部邻域进行链路预测不同,基于图路径的方法尝试利用知识图谱中的图路径信息来捕获实体与实体之间的长距离依赖。TransE-Comp和PTransE尝试建模两个实体之间的图路径上多跳关系构成的复合关系。Chain和RSN则对循环神经网络(Recurrent Neural Network, RNN)进行了改造,以学习图路径上的所有相邻实体和关系信息之间的依赖。Interstellar分析了挖掘图路径信息的重要性,并将知识图谱图路径学习问题定义为RNN神经网络架构搜索问题,并设计了一种特定于知识图谱嵌入领域的混合搜索算法以及搜索空间。

\subsection{基于Transformer的知识图谱嵌入方法}
基于图神经网络的方法通过对实体的邻域结构进行学习从而获得了阶段性的成功,性能普遍优于传统的知识图谱嵌入模型。但是图神经网络的表达能力虽然相较于传统方法的多层神经网络和卷积神经网络有了较大的提升,由于模型采用的信息聚合方式比较简单,依然存在较大的提升空间。针对这个问题,许多研究者尝试引入表达能力更强的架构。Transformer\upcite{Transformer}是注意力机制方面里程碑式的工作,基于Transformer的模型变体在计算机视觉和编程语言领域中也表现出了出色的性能,因此目前有不少工作致力于研究将Transformer结构应用到知识图谱嵌入工作中,利用Transformer强大的表达能力提高知识图谱补全的性能。根据方法原理的不同,目前基于Transformer的知识图谱嵌入方法可以大致划分为两类。

第一类方法的思路和先前的知识图谱嵌入方法类似,主要利用知识图谱本身的三元组信息来进行知识图谱嵌入任务。典型方法有HittER\upcite{HittER}和Relphormer\upcite{Relphormer}。HittER参考了图神经网络的消息传递模型,采用分层Transformer架构对实体的局部邻域进行了建模,底层Transformer模块负责完成输入信息构造,顶层Transformer负责聚合输入信息到中心实体中对语义进行学习。Relphormer提出了一种用于知识图谱嵌入的Transformer架构变体,并提出了一种Triple2Seq序列化算法来解决知识图谱中边和节点的异构性问题。这类基于Transformer的知识图谱嵌入方法最大的问题是知识图谱中的节点是无序的,需要进行一些额外的预处理或者对Transformer进行一些改造才能输入到Transformer网络中,在这个过程中容易出现结构信息的失真。此外知识图谱中的边具有不同的种类,蕴含着丰富的语义信息,如何在Transformer中对边进行准确建模也是需要考虑的问题。以Relphormer为例,模型通过路径采样的方式将实体的局部邻域转化为模型的序列输入,丧失了对局部邻域内的图结构信息的感知,而且Relphormer将实体和关系视为同类的节点进行建模,导致关系在知识图谱内的独特作用被忽略。

而由于目前的许多预训练语言模型都是基于Transformer架构来进行构建,因此另一类基于Transformer的知识图谱嵌入方法尝试通过结合预训练语言模型,利用知识图谱中的文本信息来进行知识图谱嵌入,典型方法包括KG-BERT\upcite{KG-BERT}以及LMKE\upcite{LMKE}。KG-BERT\upcite{KG-BERT}是最早利用Transformer补全进行知识图谱嵌入的方法,通过BERT\upcite{BERT}预训练语言模型来挖掘知识图谱的三元组文本中蕴含的语义信息,并基于语义信息来进行知识图谱补全。LMKE\upcite{LMKE}提出了一种对比学习框架,提高了负采样的效率,大大缩短了基于预训练语言模型的方法的训练和推理的时间,大幅度提高了补全性能。这类方法最大的缺点是基于预训练语言模型,架构复杂,参数庞大,训练和推理的成本较高,并且忽略了知识图谱的图结构信息。

\subsection{融合多源信息的知识图谱嵌入方法}

融合多源信息的知识图谱嵌入方法在以上几类算法的基础上利用更多的额外信息来进行知识图谱嵌入,例如图路径、文本描述、实体类别或者时间顺序等。这些信息能够帮助模型从不同的维度对知识图谱进行建模,提高知识图谱补全的效果。

基于图路径的方法尝试利用知识图谱中的图路径信息来捕获实体与实体之间的长距离依赖。TransE-Comp\upcite{TransE-Comp}和PTransE\upcite{PTransE}尝试建模两个实体之间的图路径上多跳关系构成的复合关系。Chain\upcite{Chain}和RSN\upcite{RSN}则对循环神经网络进行了改造,以学习图路径上的所有相邻实体和关系之间的依赖。Interstellar\upcite{Interstellar}分析了图路径信息对知识图谱嵌入的重要性,将图路径学习问题定义为循环神经网络架构搜索问题,并设计了一种特定于知识图谱嵌入领域的混合搜索算法以及搜索空间。

除了结构信息之外,知识图谱中的每个实体和关系一般都有对应的文本描述,蕴含对应的自然语言语义。NTN\upcite{NTN}对文本描述的词向量进行平均来初始化实体的向量表示。而随着最近几年预训练语言模型(Pre-trained language models, PLM)的火热发展,也有不少方法探究利用PLM来完成知识图谱补全任务。TagReal\upcite{TagReal}利用PLM结合语料库信息搜索进行知识图谱补全,并开发了自动的提示(Prompts)生成和信息检索方法,使TagReal能够自动生成高质量提示支持PLM搜索相关信息,这使得模型在PLM缺乏某些领域知识时更加实用。基于Transformer的知识图谱嵌入方法KG-BERT\upcite{KG-BERT}和LMKE\upcite{LMKE}也可以归于此类。
% Text-Augmented Open Knowledge Graph Completion via Pre-Trained Language Models

此外,部分方法还考虑了知识的时效性。TTransE\upcite{TTransE}模型在TransE的基础上进行了改进,引入了额外的时序信息;TeAST\upcite{TeAST}采用了阿基米德螺旋时间线来对时序知识图谱进行编码,将时序知识图谱的四元组补全问题转化为了三阶张量补全问题,降低了复杂度。
% TeAST: Temporal Knowledge Graph Embedding via Archimedean Spiral Timelin

总的来说,融合多源信息的知识图谱嵌入方法通过引入额外的信息获得了更好的知识图谱嵌入效果,但是往往需要额外的数据准备工作,成本较高。有些知识图谱甚至无法获取对应的信息,可移植性较差;另外数据的质量也会对模型的性能造成影响。

\subsection{对比分析}
通过调研国内外知识图谱嵌入方法,本文对于各类知识图谱嵌入方法进行了总结与对比,各种方法的优缺点如表\ref{ComparativeAnalysis}所示。
\begin{table}[tbp]
  \renewcommand\arraystretch{1.5}
  \caption{各类知识图谱嵌入方法对比分析}
  \label{ComparativeAnalysis}
  \centering
  
  \begin{tabular}{lp{6cm}p{5cm}}
    \toprule
    \multicolumn{1}{c}{方法类型} & \multicolumn{1}{c}{优点} & \multicolumn{1}{c}{缺点}\\
    \midrule
    基于翻译 & 结构简单,计算速度快,可解释性较强& 模型表达能力弱\\
    基于张量分解 & 可解释性较强,能够捕捉实体与关系之间的双线性关系 & 模型表达能力较弱\\
    基于多层神经网络 & 表达能力相比于之前的方法更强 & 容易出现过拟合的问题,嵌入的维度对性能的影响较大\\
    基于卷积神经网络 & 实体和关系之间的交互得到了增强,参数数量较少&没有利用到知识图谱的图结构信息\\
    基于图神经网络 & 能够学习到实体的局部邻域信息,模型性能相较于传统方法得到了提高 & 模型的表达能力不足以充分学习知识图谱的语义信息,另外捕获长距离信息的能力不足\\
    基于Transformer &通过自注意力机制和更复杂的网络结构获得了更强大的模型表达能力 & 模型复杂度高,不适用于大规模知识图谱,无法直接利用图结构信息\\
    融合多源信息 & 将现有方法与额外的信息进行结合,获得了更好的知识图谱嵌入效果 & 需要额外的数据准备工作,成本较高;信息的质量对模型的性能影响较大,可移植性相对较差\\
    \bottomrule
  \end{tabular}
\end{table}


本课题主要研究基于Transformer的知识图谱补全方法,因此对基于Transformer的知识图谱嵌入方法进行了进一步的对比和分析,结果如表\ref{ComparativeAnalysis_2}所示。

\begin{table}[htbp]
  \renewcommand\arraystretch{1.5}
  \caption{基于Transformer的知识图谱嵌入方法对比分析}
  \label{ComparativeAnalysis_2}
  \centering
  \begin{tabular}{lp{6cm}p{5cm}}
    \toprule
    \multicolumn{1}{c}{方法类型} & \multicolumn{1}{c}{优点} & \multicolumn{1}{c}{缺点}\\
    \midrule
    基于知识图谱三元组 &通过自注意力机制和更复杂的网络结构获得了更强大的模型表达能力 & 知识图谱中的节点是无序的,输入到Transformer中容易出现图结构信息的失真\\
    基于文本信息 & 引入了额外的文本信息,从自然语言中挖掘图谱的语义,模型的泛化能力更强 & 模型架构复杂,参数庞大,训练和推理的成本较高,并且忽略了知识图谱的图结构信息。\\
    \bottomrule
  \end{tabular}
\end{table}

\section{研究目标及内容}
\subsection{研究目标}
% 本课题的研究目标是设计基于Transformer的知识图谱嵌入模型,利用Transformer模型的强大表达能力来学习实体和关系的合适嵌入表示,对知识图谱进行自动化补全。
本文针对传统知识图谱嵌入和基于图神经网络的方法表达能力弱、图信息利用不足、无法捕获长距离信息乃至全局信息的问题,主要研究如何基于Transformer网络和知识图谱的特点,采用合适的方式采样和编码知识图谱中局部邻域和图路径两类图结构,并进行综合利用以充分发挥Transformer网络强大的表达能力,最终得到能够尽可能拟合现有图谱的合适表示。

\subsection{研究内容}


针对本文的研究目标,本文的主要研究路线如图\ref{research_route}所示。本文的研究内容主要包括以下几个方面:

\begin{figure}[htb]
  \centerline{\includegraphics[width=0.9\textwidth]{pic/research_route_2.pdf}}
  \caption{研究路线示意图}
  \label{research_route}
\end{figure}

(1)基于Transformer的模型对于知识图谱结构的捕获研究

在Transformer中,任意一个位置都能直接感知到其他位置的输入信息,这导致模型无法直接捕捉到输入之间的相对位置关系,因此在处理序列数据时,采用的方式一般是为每个位置的输入添加对应的位置编码,标识输入与输入之间的前后位置关系。但在知识图谱中节点并不是顺序排列的,因此本文的主要研究内容之一就是设计一种合适方案让Transformer模型能够学习到知识图谱的拓扑结构,实现对知识图谱结构的感知。此外,在知识图谱中关系也是重要的结构信息,反映着实体和实体之间不同的交互方式,蕴含着丰富的语义信息,但标准的Transformer模型并没有直接的方式来编码和使用边的信息,因此Transformer如何对边信息进行表示也是重要的研究内容。

(2)图路径采样算法研究

针对基于图神经网络的方法对知识图谱中长距离信息学习不足的缺陷,本文计划通过对知识图谱中的图路径信息学习来挖掘实体与实体之间的长距离依赖,因此为了提升模型性能,对于当前的待遇测事实三元组,如何采样到高质量的图路径是首先需要解决的问题。因此本文计划研究设计合适的采样策略,实现高效的图路径采样,提高模型捕获长距离依赖的能力。

(3)不同图结构信息的结合方案研究

基于图神经网络的模型通过聚合消息的方式实现了对于中心实体局部邻域结构的感知,但无法捕捉实体之间长距离的依赖;基于图路径的方法能够挖掘到更远距离的依赖,但忽略了实体丰富的局部邻域。因此,本文的主要研究内容之一是设计合适的模型结构实现对于以上两类图结构信息的综合利用,实现对于图谱中长短距离信息的捕捉。

(4)实验与验证

在完成以上研究内容,实现完整的知识图谱补全模型之后,设计相应实验方案,通过平均排名、平均倒数排名等指标在主流公开数据集上与基线模型进行性能对比,验证本文提出的模型的有效性;并且通过设计合适的消融实验,验证模型关键设计的有效性。

\subsection{论文组织安排}
本文对基于Transformer的知识图谱补全方法进行研究,论文内容总共分为五个章节以及总结与展望部分,各个章节的内容安排组织如下:

第一章绪论首先介绍了论文的背景与意义,随后对知识图谱以及知识图谱算法的国内外研究进展进行了简单介绍,并进行了各类方法的对比与总结。随后明确了论文的研究目标与研究路线,概述了论文的主要研究内容。最后介绍了论文的组织安排。

第二章介绍了论文中方法所涉及到的相关理论基础。首先介绍了注意力机制与Transformer模型架构,其次对Transformer模型在图学习领域中的应用进行了概述,最后对不同类别的知识图谱嵌入方法进行了介绍,包括部分模型的核心思想以及数学公式。

第三章首先对Transformer模型在知识图谱嵌入领域的应用存在的困难进行了分析,随后介绍了提出的基于邻域感知Transformer的知识图谱补全模型,给出了符号定义以及模型的总体架构,并对其中的关键设计结构强化的自注意力机制进行了说明。

第四章首先指出了现有的基于图结构信息的方法的缺点,随后介绍了提出的结合图路径和局部邻域的Transformer知识图谱嵌入模型,说明了模型的总体架构以及各个模块的设计方案,包括基于有偏随机游走的图路径采样算法、图路径信息与局部邻域信息的结合方案以及掩蔽实体关系预测任务。

第五章为实验与验证部分,首先说明了实验采用的数据集、选取的进行对比的基线模型、实验环境以及采用的评估策略等基本情况,随后对实验结果进行了介绍和分析,包括本文提出的模型与基线模型的总体性能对比、关键模块消融实验的结果,超参数对于模型的性能影响等。

总结与展望部分对本文的研究内容进行了回顾与总结,并对未来可能的研究方向进行了展望。

\chapter{相关理论基础}
本章对论文中所涉及到的相关理论基础进行了介绍。首先介绍了注意力机制与Transformer模型架构,其次对Transformer模型在图表示学习领域中的应用进行了概述,最后对不同类型的知识图谱嵌入方法的核心思想和数学公式进行了说明,包括传统的知识图谱嵌入方法、基于图神经网络的知识图谱嵌入方法、基于图路径的知识图谱嵌入方法以及基于Transformer的知识图谱嵌入方法。

\section{注意力机制与Transformer网络}
深度学习中的注意力机制(Attention Mechanism)灵感来源于人类的视觉和认知系统。在推理过程中,注意力机制动态的为输入数据分配不同的权重,使模型能够自动地学习并选择性地关注输入中的重要信息,提高模型的性能和泛化能力。注意力机制最早被用于处理计算机视觉任务,后来在多个领域中得到了应用,例如自然语言处理和推荐系统等。

谷歌的研究团队于2017年提出的Transformer网络则是注意力机制方面里程碑式的工作。Transformer网络设计之初主要用于处理序列数据,在Transformer出现之前,序列数据的处理通常依赖于循环神经网络(RNN)及其变体,例如长短期记忆网络(LSTM)和门控递归单元(GRU)。RNN及其变体在处理序列数据时能够保持一定程度的历史信息,但存在一定的问题:由于对序列数据进行逐步处理,RNN在训练过程中容易出现梯度消失或者梯度爆炸的问题,特别是在处理长序列时;逐步处理也限制了模型的并行计算能力,导致训练效率低下;此外,尽管LSTM和GRU通过特殊的门控机制改善了长距离依赖问题,但当序列长度过高时,模型依然难以捕捉到距离较远的依赖关系。

Transformer网络通过使用自注意力(Self-Attention)机制解决了上述问题,在自注意力机制中,输入数据中的任意一个位置都能够直接感知到其它位置的信息,因此相比于传统的RNN结构,自注意力能够更加直接地捕捉到序列中长距离的依赖关系;自注意力机制允许模型在处理数据时并行计算各个位置的注意力分数,与RNN逐步计算的方式相比,可以显著提高模型的计算效率;自注意力机制通过学习输入序列中不同位置之间的动态相关性,能够根据特定的任务自适应地调整注意力分布。

Transformer网络中的自注意力机制的核心为缩放点积注意力机制,结构如图\ref{DotProductAttention}所示。
\begin{figure}[htbp]
  \centerline{\includegraphics[width=0.30\textwidth]{pic/DotProductAttention.png}}
  \caption{缩放点积注意力机制}
  \label{DotProductAttention}
\end{figure}

具体来说,假设模型的输入为$X$,首先模型将会通过线性变化生成输入对应的查询向量$Q$,键向量$K$以及值向量$V$:
\begin{gather}
    Q=XW^Q, K=XW^K, V=XW^K
\end{gather}
其中$W^Q$、$W^K$、$W^K$是可学习的参数矩阵。

随后模型会将查询向量$Q$和键向量$K$进行点积并乘以缩放因子$\frac{1}{\sqrt{d_k}}$获得注意力分数,将其进行归一化处理转化为概率分布,用作权重对值向量$V$进行加权平均和,最终得到缩放点积注意力机制对应的输出,其中$d_k$为键向量的维度:
\begin{gather}
    \mbox{Attention}(Q,K,V) = \mbox{softmax}(\frac{QK^T}{\sqrt{d_k}})V
\end{gather}


进一步的,为了让模型能够同时关注来自不同维度的信息,并稳定自注意力的学习过程,Transformer采用了多头注意力机制,通过不同的参数矩阵将$Q$、$K$、$V$映射到不同的向量空间下并计算缩放点积注意力,将结果进行拼接获得最终的输出,如图\ref{MultiHeadAttention}所示。

具体来说,对于$h$个独立的注意力头,有:
\begin{equation}
  \begin{aligned}
    \mbox{MultiHead}(Q,K,V)&=\mbox{Concat}(head_1,...,head_h)W^O\\
    \mbox{where} \  head_i &=\mbox{Attention}(QW^Q_i,KW^K_i,VW^V_i)
  \end{aligned}
\end{equation}
其中$QW^Q_i$、$KW^K_i$、$VW^V_i$为将$Q$、$K$、$V$映射到第$i$个向量空间的参数矩阵,Concat为拼接操作。

\begin{figure}[htbp]
  \centerline{\includegraphics[width=0.4\textwidth]{pic/MultiHeadAttention.png}}
  \caption{多头注意力机制}
  \label{MultiHeadAttention}
\end{figure}
\section{基于Transformer的图表示学习方法}

\section{知识图谱嵌入方法}

\subsection{传统的知识图谱嵌入方法}

\subsection{基于图神经网络的知识图谱嵌入方法}

\subsection{基于图路径的知识图谱嵌入方法}

\subsection{基于Transformer的知识图谱嵌入方法}

\chapter{基于邻域感知的Transformer模型}


% 本章主要对基于邻域感知的Transformer模型NATLP的总体设计和模块的具体实现进行了介绍。主要包括对现存基于图神经网络方法存在问题的分析、模型的总体框架设计、关系特定的邻居实体信息构造设计以及融合图结构信息的自注意力机制的改进。

相较于传统的知识图谱嵌入方法,基于图神经网络的知识图谱嵌入方法通过聚合局部邻域信息获得了很大的性能提升。但是图神经网络采用的聚合方式忽略了邻居节点信息之间的潜在关联,并且模型结构较为简单。针对以上问题,本文尝试利用Transformer模型来提高模型的表达能力,完成知识图谱补全任务,提出了一种基于邻域感知的Transformer模型用于链路预测任务(Neighborhood Aware Transformer for Link Prediction, NATLP)。
% Transformer网络虽然在自然语言处理(NLP)领域取得了巨大成功,但在直接应用到图表示学习(Graph Representation Learning)领域时会遇到一系列挑战和困难。图表示学习涉及到处理图数据,图数据具有非欧几里得结构,这与NLP中处理的序列数据(一维结构)有本质区别。下面列举了一些Transformer在图表示学习中面临的主要困难:

% 1. **非序列结构**:图是一种无序的数据结构,节点之间的关系是通过边来定义的,并且没有固定的顺序。相比之下,Transformer网络原本是为处理序列数据设计的,它使用位置编码来保留序列中元素的顺序信息。在图中,节点没有自然的顺序,因此Transformer网络无法直接使用位置编码来捕捉节点间的结构关系。

% 2. **邻居多样性**:在图中,每个节点可能有不同数量的邻居。这种动态的、不规则的邻接关系与Transformer网络中假设的固定大小的输入上下文不相符。因此,直接应用自注意力机制来学习图节点表示时难以处理不同节点的可变邻居数量。

% 3. **长距离依赖**:尽管Transformer网络擅长捕捉长距离依赖,但在图中,这种依赖通常是通过节点之间的路径来定义的,而不是序列中的线性位置。因此,Transformer需要被修改或扩展,以便能够理解和利用图中的路径和结构模式。

% 4. **缩放问题**:对于大型图,使用Transformer进行全图的自注意力学习是计算资源密集型的,因为复杂度是图中节点数量的平方。这使得直接在大规模图上应用Transformer变得不切实际。

% 5. **边的信息和类型**:在图中,边可能包含重要的信息,并且边的类型(如社交网络中的不同关系类型)也可能对图表示至关重要。然而,标准的Transformer模型并没有直接的方式来编码和使用边的信息。

% 为了克服这些困难,研究人员已经提出了许多不同的方法和模型的变种,如图注意力网络(Graph Attention Networks, GATs)和图Transformer网络(Graph Transformer Networks),这些模型专门设计或修改以适应图数据的特性。这些变种通常包括对自注意力机制的修改,以更好地处理图的结构性质,以及引入适合于图结构的新型位置编码或相对位置编码。此外,为了处理大型图,研究人员还在探索如何有效地对图进行采样,以及如何利用稀疏性和分层结构来降低计算复杂度。

% Transformer在知识图谱嵌入任务中具有以下优点:

% 1. 处理长距离依赖关系:知识图谱中的实体和关系之间存在着复杂的长距离依赖关系。传统的基于图的模型(如Graph Convolutional Networks)往往只考虑局部邻居信息,难以捕捉到远距离的语义依赖。而Transformer通过自注意力机制可以处理全局上的依赖关系,能够更好地捕捉实体和关系之间的长距离语义关联。

% 2. 并行计算能力:与循环神经网络不同,Transformer网络可以进行并行计算,加快了训练和推理的速度。在大规模的知识图谱中,由于实体和关系的数量庞大,传统的基于图的模型可能面临计算效率低下的问题,而Transformer网络能够更高效地处理大规模的知识图谱数据。

% 3. 对复杂属性建模:知识图谱中的实体通常具有丰富的属性信息,如文本描述、分类标签等。传统的基于图的模型往往无法有效地捕捉和利用这些属性信息,而Transformer网络可以将这些属性信息作为输入,并通过多头注意力机制对不同属性进行建模,从而更好地捕捉实体的语义特征。

% 4. 高度可扩展性:Transformer网络可以通过增加层数和调整参数来提高模型的表示能力。在知识图谱嵌入任务中,复杂的语义关系和多层次的结构需要一个具有较强建模能力的模型来学习。Transformer网络的多层结构可以提供更强的表示能力,可以通过增加层数来适应不同任务的需求。

% 综上所述,Transformer在知识图谱嵌入任务中具有处理长距离依赖关系、并行计算能力、对复杂属性建模和高度可扩展性等优点,为知识图谱的表示学习提供了一种强大而灵活的模型选择。

%注意力中添加度数:1-1/度数乘积

\section{现有问题描述和分析}

图卷积神经网络GCN于2017年被提出,对原先图神经网络中基于谱空间的的图卷积算子进行了优化,降低了模型的复杂度,由此引发了图神经网络的研究热潮。图神经网络迅速成为了图结构数据处理的重要方式,在社交网络\upcite{social}、推荐系统、知识图谱等多个领域都有着重要应用。近几年,图神经网络的应用是知识图谱嵌入领域非常重要的进展。图卷积神经网络能够直接处理图结构数据并捕捉知识图谱中的拓扑结构,通过聚合邻居节点的信息,图神经网络可以有效地学习知识图谱中节点(实体)和边(关系的嵌入表示)。相较于之前的方法,基于图神经网络的知识图谱嵌入方法获得了很大的性能提升。

然而,受限于本身的网络结构,在进行知识图谱嵌入时,基于图神经网络的方法依然存在不足,导致其性能受限。首先,图卷积神经网络采用聚合邻居节点的方法来更新中心实体的表示,在这个过程中,模型只考虑了邻居节点和中心实体之间的连通性,却忽略了不同邻居节点之间的可能也存在直接连接,各个邻居节点传递的信息之间是互不感知、互相独立的,这样的聚合方式没有将邻居节点信息之间的相互依赖纳入考虑;其次,图神经网络采用的消息传递模式整体模型结构比较简单,使模型的表达能力受到了限制,在挖掘图谱中实体和实体、实体与关系之间的复杂交互上存在困难。

而在以上两个方面,Transformer网络存在巨大的优势。首先,通过构造查询向量、键向量和值向量来进行注意力的计算以及采用多头注意力机制,相比于图神经网络,Transformer能够更加高效地挖掘输入之间各个维度的复杂交互;同时,通过调整模型的层数、头的数量或是隐层的维度大小,Transformer可以很容易地适应处理不同规模和复杂度的知识图谱的需求。

其次,Transformer网络的自注意力机制能够有效地捕捉序列中任意两个元素之间的全局依赖关系。在知识图谱嵌入的场景之中,这意味着模型在挖掘局部邻域的结构信息时,除了邻居节点和中心节点之间的依赖之外,还能够同时学习到邻居节点之间的长距离依赖,捕获邻居节点传递的信息之间的相互影响。此外,Transformer架构还支持模型的预训练和迁移学习,可以首先在一个大规模的综合性知识图谱上进行预训练,然后迁移到特定领域的知识图谱上,通过这样的方式,可以减少模型对标注数据的依赖,提高模型在特定任务上的表现。

虽然Transformer网络在自然语言处理(NLP)领域已经取得了巨大成功,但将Transformer网络直接应用到知识图谱嵌入领域时依旧会遇到一系列挑战和困难。图具有非欧几里得结构,是一种无序的数据结构,这与NLP中处理的序列数据(一维结构)有本质区别。Transformer网络原本是为处理序列数据设计的,它使用位置编码来保留序列中元素的顺序信息。但是,在知识图谱中,节点之间的关系是通过边来定义的,并且没有固定的顺序,因此Transformer网络无法直接使用位置编码来捕捉节点间的结构关系。部分方法例如MAGNN\upcite{MAGNN}选择通过随机游走的方式来将图数据转化为序列数据来处理,但这样的方式会导致图结构信息的失真。

此外,在知识图谱中,边,即关系,反映着实体和实体之间不同的交互方式,蕴含着丰富的语义信息。两个实体之间连接的关系不同,传递的信息可能是千差万别的,因此如何采用合适的方式来对利用关系信息,体现关系对于消息传递的影响十分重要。但标准的Transformer模型并没有直接的方式来编码和使用边的信息。部分利用Transformer来进行知识图谱表示学习的模型例如Relphormer\upcite{Relphormer}将知识图谱中的实体和关系视为地位相同的节点,采用同样的方式进行处理,这样的方式虽然解决了边的表示问题,但没有考虑到知识图谱中实体和关系的差异性,没有考虑到关系对于实体消息传递的独特作用。

为了解决上述问题,本章提出了一种基于邻域感知的Transformer模型用于链路预测任务(Neighborhood Aware Transformer for Link Prediction, NATLP)。首先,在模型输入信息构造阶段,为了充分建模不同关系对于实体传递消息的影响,模型基于关系生成特定的网络参数,实现关系特定的邻居信息构造。其次,为了让Transformer能够更好地处理图结构数据,模型对Transformer的自注意力机制进行了改造,提出了一种融合图结构的自注意力机制,使得Transformer能够学习到输入消息之间的互相依赖。

\section{NATLP模型设计}

\subsection{符号定义}

为了方便说明论文提出的NATLP模型的实现细节,本节首先对模型中的关键概念和相关的数学符号进行了定义,具体内容参见表\ref{definition}。

\setlength{\tabcolsep}{20pt}

\renewcommand\arraystretch{1.2}
\begin{longtable}[htbp]{cc}
  % 首页表头
  \caption{NATLP模型中的符号定义}
  \label{definition}\\
  \toprule
  符号  & 说明\\
  \midrule
  \endfirsthead
  % 续页表头
  \caption{NATLP模型中的符号定义}\\
  \toprule
  符号  & 说明 \\
  \midrule
  \endhead
  % 首页表尾
  \hline
  % \multicolumn{2}{r}{\small 续下页}
  \endfoot
  % 续页表尾
  \bottomrule
  \endlastfoot
  
  $\mathcal{G}$   &   知识图谱      \\
  $\mathcal{E}, \mathcal{R}, \mathcal{T}$   &   实体集合、关系集合、边集合      \\
  $\mathcal{G}^\prime$  &  拓展后的知识图谱      \\
  $\mathcal{R}^{\prime}$   &   拓展后的关系集合      \\
  $\mathcal{T}^{-1}$   &   逆关系边集合      \\
  $\mathcal{T}^{\prime}$   &   拓展后的边集合      \\
  $(s,r,?)$  &   待遇测的三元组      \\
  $s$   &   头实体即中心实体      \\
  $o$   &   尾实体即目标实体      \\
  $e$   &   实体      \\
  $r$   &   关系      \\
  $r^{-1}$   &   关系$r$的逆关系      \\
  $\boldsymbol{s},\boldsymbol{o}$ & 头实体嵌入和尾实体嵌入\\
  $\boldsymbol{e},\boldsymbol{r}$ & 实体嵌入和关系嵌入\\
  $d$ &嵌入维度\\
  $\phi_{chk}$ & 棋盘式特征重组\\
  $\circledast$ & 循环卷积操作\\
  $f(\cdot )$ & ReLU激活函数\\
  $vec(\cdot)$ & 二维张量转化为一维向量\\
  $k_{size}$ &卷积核的边长\\
  $n_{conv}$ &卷积核的数量\\
  $\omega_r$ & 特定于关系$r$的卷积层参数\\
  $\mathbf{W}_r$ &特定于关系$r$的全连接层参数\\
  $\mathbf{W}_{conv}$ & 卷积层参数生成网络\\
  $\mathbf{W}_{fc}$ & 全连接层参数生成网络\\
  $\boldsymbol{r}_{global}$ & 全局关系嵌入\\
  $\mbox{Re3D}(\cdot)$ & 一维向量转化为三维张量\\
  $\mbox{Re2D}(\cdot)$ & 一维向量转化为二维张量\\
  $\boldsymbol{m}_{e,r}$ & 邻居实体$e$通过关系$r$传递的信息\\
  $\boldsymbol{e}_{cls}$ & 特殊嵌入Class Token\\
  $\mathbf{TE}$ & 类型嵌入\\
  $a_{ij}$ & 第i个输入和第j个输入之间的注意力得分\\
  $dis(e_i,e_j)$ & 实体$e_i$与实体$e_j$之间最短路径的距离\\
  $deg(e)$ & 实体$e$的节点度数\\
  $\boldsymbol{o}_t$ & 模型预测的候选实体的嵌入\\
  $\ast $ & 普通卷积操作\\
  $\sigma $ & sigmoid激活函数\\
  $p$ & 三元组正确概率\\
  $L$ & 模型损失\\
  $t_i$ & 第i个三元组的标签\\

\end{longtable}

知识图谱是表示为$(s,r,o)$的“头实体-关系-尾实体”事实三元组的集合,所有的这些事实三元组连接起来构成了一个异构图,即为知识图谱,表示为$\mathcal{G} = (\mathcal{E}, \mathcal{R}, \mathcal{T})$,其中$\mathcal{E}$集合,$\mathcal{R}$为关系集合,$\mathcal{R}$为实体和关系构成的边集合。由于知识图谱中的关系具有方向性,为了确保信息能够在两个相连的实体之间进行双向的流通,本文在知识图谱中为每个事实三元组$(s,r,o)$创建了对应的逆三元组$(o,r^{-1},s)$,其中$r^{-1}$是关系$r$对应的逆关系。因此,关系集合、边集合以及知识图谱被拓展为:
\begin{gather}
    \mathcal{R}^{\prime}=\mathcal{R}\cup\{ r^{-1} | r\in \mathcal{R}\}\\
    \mathcal{T}^{-1}= \{ (o,r^{-1},s)| (s,r,o)\in \mathcal{T}\}\\
    \mathcal{T}^{\prime} = \mathcal{T}\cup\mathcal{T}^{-1}\\
    \mathcal{G}^\prime = (\mathcal{E}, \mathcal{R}^\prime, \mathcal{T}^\prime)
\end{gather}

知识图谱补全任务,即链路预测任务,是在给定待遇测三元组中的头实体$s$以及关系$r$的情况下预测缺失的尾实体$o$,表示为$(s, r, ?)$,或者是在给定尾实体$o$以及逆关系$r^{-1}$的情况下预测缺失的头实体$s$,表示为$(?, r^{-1} , o) $。为了方便说明,论文随后统一采用$(s, r, ?)$的形式进行表述。


\subsection{模型总体结构}

本节主要对提出的用于链路预测的基于邻域感知的Transformer模型NATLP的总体结构进行介绍。NATLP整体为编码器-解码器架构,编码器部分主要由关系特定的邻居信息构造模块和邻域感知Transformer模块组成,解码器部分则采用了基于卷积神经网络的知识图谱嵌入方法进行了实现。模型整体架构如图\ref{architecture}所示。

\begin{figure}[htb]
  \centerline{\includegraphics[width=1\textwidth]{pic/architecture.pdf}}
  \caption{NATLP模型整体架构}
  \label{architecture}
\end{figure}

编码器的主要作用是将模型输入的实体和关系转化为对应的嵌入,并学习其中蕴含的语义信息和结构信息并编码成向量形式,是模型的核心部分。在NATLP中,模型的输入主要包括待预测的三元组及其局部邻域,编码器首先会根据中心实体和邻居之间相连的关系种类,为每一个邻居实体构造关系特定的邻居消息;随后的邻域感知Transformer模块综合学习构造的邻居信息、中心实体本身的信息以及局部邻域的结构信息,并完成编码。

解码器的主要任务则是根据编码器得到的知识表示,对下游任务的各项性能指标进行评测。根据下游任务的不同,模型可以采用不同的解码器进行解码。NATLP采用了基于卷积神经网络的知识图谱嵌入方法ConvE作为解码器,来对事实三元组的正确概率进行评估,完成知识图谱补全。

\subsection{关系特定的邻居实体信息构造}

链路预测任务的目标是利用知识图谱中已有的事实去预测未知事实在知识图谱中的存在概率。为了能够充分利用邻域信息来帮助预测三元组中缺失的尾实体,NATLP需要获得局部邻域中邻居实体向中心实体传递的信息。知识图谱中的关系反映着实体和实体之间不同的交互方式,但标准的Transformer模型没有办法直接对关系进行编码。为了解决这个问题,受到基于图神经网络的知识图谱嵌入方法中的消息传递模型的启发,NATLP首先基于连接的关系完成邻居实体的消息构造后,再将消息传递到Transformer模型中进行学习。

但是,目前基于图神经网络的知识图谱嵌入方法中采用的消息构造函数存在着一些不足。本文调研了部分基于图神经网络的知识图谱嵌入方法采用的消息构造函数,具体内容见表\ref{message_Function}。

\begin{table}[htbp]
  \renewcommand\arraystretch{1.5}
  \caption{部分基于图神经网络的知识图谱嵌入方法采用的消息构造函数}
  \label{message_Function}
  \centering
  \begin{tabular}{cc}
    \toprule
    知识图谱嵌入方法 & 采用的消息构造函数\\
    \midrule
    R-GCN\upcite{R-GCN} & $\mathbf{W}_re$\\
    SACN\upcite{SACN} & $\mathbf{W}e$\\
    Graph2Seq\upcite{Graph2Seq} & $\mathbf{W}_{in}\left[e_i,r_k,e_j\right] \ \mbox{or} \ \mathbf{W}_{out}\left[e_i, r_k, e_j\right]$\\
    CompGCN\upcite{CompGCN} & $\mathbf{W}_{dir(r)}e_i\star e_k$\\
    KBGAT\upcite{KBGAT} & $\mathbf{W}\left[e_i, e_j, r_k\right] $\\
    RGHAT\upcite{RGHAT} & $\mathbf{W}_2\left[\mathbf{W}_1\left[e_i, r\right], e_j\right]$\\
    \bottomrule
  \end{tabular}
\end{table}

可以发现,除了R-GCN\upcite{R-GCN}之外,其余的方法对于实体通过不同关系传递的信息,采用的都是同样的网络参数进行编码。但是,同一个实体和不同的关系相连,表达的语义信息可能完全不一样。例如(姚明,出生于,上海)和(姚明,职业,篮球运动员),传递的信息就有着很大不同。采用同样的参数进行编码,会导致模型难以捕获实体中的和不同关系相关的特定特征。针对关系的这个特点,R-GCN模型为每个关系都定义了单独的网络参数,但是这样的方法也存在问题:一方面,每类关系的网络参数需要单独进行学习,对于数量较少的关系可能会出现训练不充分的情况;另一方面,这样的方法会容易导致关系之间的内在相关性被忽略。TransCoRe\upcite{TransCoRe}对TransE/TransH/TransR学习到的关系嵌入进行了分析,发现关系之间的相关性通过嵌入表示上的低秩结构显示出来,即不同种类的关系之间存在某种共同的特点。

为了解决上述问题,实现捕获邻居实体中关系相关的特定特征的同时,兼顾不同类别关系之间的共通特征,NATLP提出了一种关系特定的邻居实体信息构造方法,具体如图\ref{build_information}所示。

\begin{figure}[htb]
  \centerline{\includegraphics[width=0.9\textwidth]{pic/build_information.pdf}}
  \caption{关系特定的邻居实体信息构造}
  \label{build_information}
\end{figure}

首先,相比于大多数方法采用的实体和关系嵌入拼接之后再线性转换的构造方式,NATLP模型选择将实体和嵌入重塑为二维张量之后再对其进行卷积操作。相比于线性转换,卷积神经网络更擅长捕捉局部模式,通过卷积操作,模型可以有效地提取实体和关系之间的局部交互特征;此外,由于权重共享的特性,卷积神经网络在模型参数上更加高效,使得模型能够用更少的参数完成信息的构建,减少模型过拟合的风险,加快模型训练的过程。而相比于一维卷积,二维卷积能够提升实体和关系嵌入之间的特征交互,从而更丰富的特征。为了进一步的提升实体和关系之间的交互,NATLP采用了棋盘式的特征重组方式来进行二维张量的重塑,实体中的每一个特征分量能够和四个关系的特征分量进行交互,如图\ref{cross_conv}所示:

\begin{figure}[htb]
  \centerline{\includegraphics[width=0.8\textwidth]{pic/cross_conv.pdf}}
  \caption{棋盘式特征重组}
  \label{cross_conv}
\end{figure}
% NATLP采用卷积神经网络来完成邻居信息的构造,利用特定关系参数生成网络来为不同关系生成对应的卷积神经网络参数。为了显示的捕捉不同关系之间的共性,NATLP引入了一个全局关系嵌入参与网络参数的生成。

具体的,给定一个邻居实体$e$和相连的关系$r$,模型先采用棋盘式特征重组的方式将实体嵌入和关系嵌入重塑为二维张量:
\begin{equation}
  \phi_{chk}\left(\boldsymbol{e},\boldsymbol{r}\right) 
\end{equation}
其中$\phi_{chk}$代表棋盘式特征重组,$\boldsymbol{e} \in \mathbb{R} ^d$为实体$e$的嵌入表示,$\boldsymbol{r} \in \mathbb{R}^d$为关系$r$的嵌入表示,$d$为嵌入的维度。

将实体和关系嵌入表示重塑为二维张量之后,NATLP会对其进行循环卷积操作:
\begin{equation}
  \phi_{chk}\left(\boldsymbol{e},\boldsymbol{r}\right) \circledast \omega_r
\end{equation}

其中$\circledast$代表循环卷积。相比于普通的卷积操作,循环卷积能够捕捉更多的特征交互。卷积完成后,NATLP将卷积的输出重组为一维向量,再经过线性变化之后就可以得到邻居实体向中心实体传递的信息$\boldsymbol{m}_{e,r} \in \mathbb{R} ^d$:
\begin{equation}
  \boldsymbol{m}_{e,r}=f\left(vec\left(f\left(\phi_{chk}\left(\boldsymbol{e},\boldsymbol{r}\right) \circledast \omega_r \right)\right)\mathbf{W}_r\right)
\end{equation}
其中$f(\cdot )$代表ReLU激活函数,$vec(\cdot)$代表将卷积的输出重整为一维向量的操作。

为了让模型能够充分捕获关系对于实体消息传递的影响,NATLP采用参数生成网络来为卷积层和全连接层生成关系对应的特定参数$\omega_r$和$\mathbf{W}_r$。同时,为了显示地捕捉不同关系之间的共性,NATLP引入了一个全局关系嵌入$\boldsymbol{r}_{global}$参与网络参数的生成:
\begin{gather}
  \omega_r = \mbox{Re3D}\left(\mathbf{W}_{conv}\left[\boldsymbol{r};\boldsymbol{r}_{global}\right]\right) \\
  \mathbf{W}_r = \mbox{Re2D}\left(\mathbf{W}_{fc}\left[\boldsymbol{r};\boldsymbol{r}_{global}\right]\right)
\end{gather}

其中$\mathbf{W}_{conv} \in \mathcal{R}^{n_{conv}\times k_{size} \times k_{size} \times d}$为卷积层参数生成网络,$n_{conv}$为卷积核的数量,$k_{size}$为卷积核的边长,$\mathbf{W}_{fc}$为全连接层参数生成网络,$\mbox{Re3D}(\cdot)$和$\mbox{Re2D}(\cdot)$代表将参数生成网络的输出重整为卷积层和全连接层参数需要的三维张量和二维张量的形式。$\boldsymbol{r}_{global}$为全局关系嵌入。通过全局关系嵌入$\boldsymbol{r}_{global}$,模型能够捕捉到不同关系之间的共同特征,当部分关系种类训练数据较少时,模型通过全局关系嵌入也能获得不错的泛化能力。

\subsection{邻域感知Transformer模块}

完成关系特定的邻居实体信息构造之后,NATLP下一步的任务是利用Transformer模型来挖掘局部邻域蕴含的语义和结构信息,并编码成向量形式,提供给解码器进行解码。

给定一个待遇测的事实三元组$(s,r,?)$,其中头实体即中心实体$s$有$n$个邻居实体,$e_i$、$r_i$为中心实体的邻居实体和对应的相连的关系,有$\forall  i\in\left[1,n\right] $,$(s,r_i,e_i) \in \mathcal{T}^\prime $,则在完成关系特定的邻居实体信息构造之后,邻域感知Transformer模块的输入可以表示为以下形式:
\begin{gather}
  \mathbf{M}_{input}=\left[\boldsymbol{s},\boldsymbol{m}_{e_1,r_1},\boldsymbol{m}_{e_2,r_2},...,\boldsymbol{m}_{e_n,r_n}\right] \\
  \varPhi \left(e,r\right) = f\left(vec\left(f\left(\phi_{chk}\left(\boldsymbol{e},\boldsymbol{r}\right) \circledast \omega_{r} \right)\right)\mathbf{W}_{r}\right)\\
  \boldsymbol{m}_{e_i,r_i}=\varPhi \left(e_i,r_i\right)
\end{gather}
其中$s$为中心实体的嵌入表示,$\boldsymbol{m}_{e_i,r_i}$为实体$e_i$通过关系$r_i$传递的信息。

此外,为了防止在解码的时候模型对某个特定的输入具有偏向性,模型在输入序列的头部添加了一个特殊的嵌入Class Token,表示为$\boldsymbol{e}_{cls}$。Class Token不基于任意的输入内容,在训练之前进行随机的初始化,并且随着网络的训练不断更新,能够在一定程度上编码整个知识图谱的统计特性。最终,在Transformer的输出中,Class Token对应的输出向量被用作代表整个输入序列的特征表示,传递给解码器。为了帮助模型在Class Token、中心实体嵌入表示和邻居实体传递的消息之间进行区分,受到BERT\upcite{BERT}模型的启发,NATLP为以上三类输入分配了可学习的类型嵌入,则邻域感知Transformer模块的最终输入可以表示为:
\begin{gather}
  \mathbf{M}_{input}^\prime = \left[\boldsymbol{e}_{cls},\boldsymbol{s},\boldsymbol{m}_{e_1,r_1},\boldsymbol{m}_{e_2,r_2},...,\boldsymbol{m}_{e_n,r_n}\right]\\
  \mathbf{M}_{input} = \mathbf{M}_{input}^\prime + \mathbf{TE}
\end{gather}
其中$\mathbf{TE}$代表可学习的类型嵌入。

在完成输入的构造之后,NATLP将利用Transformer的自注意力机制来学习输入中的信息。原始版本的Transformer的第$i$个输入和第$j$个输入之间的注意力分数$a_{ij}$计算公式为:
\begin{equation}
  \label{attention_score}
  a_{ij}=\frac{(\boldsymbol{m}_iW_Q)(\boldsymbol{m}_jW_K)^T}{\sqrt{d}}
\end{equation}
这样的计算方式给Transformer带来的最大优势是让其具备了捕捉输入中全局信息的能力。在Transformer的每一层中,所有的输入都能够接收并处理来自输入序列中任何位置的信息。然而,这样的方式也带来了副作用:输入序列中的结构信息丢失了,在处理邻居实体传递的信息时,模型无法捕捉到邻居实体之间的直接联系,因此模型需要想办法明确区分不同的位置信息或者分辨不同输入之间的位置相关性。在处理序列数据时,可以采用为不同位置的输入分配不同的位置向量的方法解决这个问题,但这种方法并不适合知识图谱这种非欧结构的数据。

为了让Transformer能够捕获中心实体局部邻域中的图结构数据,本文提出了一种节点距离编码,模型根据节点之间在图谱中的最短距离来辅助计算输入之间的注意力分数。一般来说,实体应该更关注距离较近的其他实体。具体来说,当计算注意力时,模型额外添加一个基于实体节点间最短距离的偏置项:
\begin{equation}
  a_{ij}=\frac{(\boldsymbol{m}_iW_Q)(\boldsymbol{m}_jW_K)^T}{\sqrt{d}}+\frac{1}{dis(e_i,e_j)}
\end{equation}
其中$dis(e_i,e_j)$为知识图谱中实体$e_i$和$e_j$之间的最短路径的距离。通过添加额外的辅助项,距离越近的实体之间计算得到的注意力得分越高。

此外,在公式\ref{attention_score}中注意力分数是基于输入信息之间的语义相关性计算的,但是知识图谱中实体的节点度数也是重要的结构信息,它衡量了实体在知识图谱中的重要性。例如,在社交网络知识图谱中,拥有大量关注者的明星的权重应该更高。因此在注意力计算中,节点度数也应该被纳入考虑。具体来说,NATLP在注意力计算中额外添加一个节点度数的辅助项:
\begin{equation}
  a_{ij}=\frac{(\boldsymbol{m}_iW_Q)(\boldsymbol{m}_jW_K)^T}{\sqrt{d}}+1-\frac{1}{\lg (deg_{e_i})\cdot \lg (deg_{e_j})}
\end{equation}
其中$deg_{e_i}$为实体$e_i$的节点度数。两个实体的节点度数越高,注意力得分越高。通过这样的方式,模型能够在注意力机制中同时捕获语义相关性和节点的重要性。

最终,领域感知Transformer模块的自注意力计算方式为:
\begin{equation}
  a_{ij}=\frac{(\boldsymbol{m}_iW_Q)(\boldsymbol{m}_jW_K)^T}{\sqrt{d}}+\frac{1}{dis(e_i,e_j)}+1-\frac{1}{\lg (deg_{e_i})\cdot \lg (deg_{e_j})}
\end{equation}
模型取Transformer最后一层的输出中Class Token对应位置的输出向量$\boldsymbol{T}_{cls}$作为整个编码器的最终输出。


\subsection{基于卷积神经网络的解码器}

解码器的主要任务是根据邻域感知Transformer模块的输出来计算待遇测三元组正确的概率,对链路预测任务的效果进行评估。在知识图谱补全任务中,一般采用传统的知识图谱嵌入方法作为解码器,它们结构简单,计算效率高,可解释性强。常见的解码器有基于翻译的方法如TransE\upcite{TransE}、基于张量分解的方法DistMult\upcite{DistMult}以及基于卷积神经网络的方法ConvE\upcite{ConvE}。在这之中性能最好,最常被使用的解码器是ConvE,因此NATLP也采用ConvE作为解码器。

给定待遇测的事实三元组$(s,r,?)$和编码器的输出$\boldsymbol{T}_{cls}$,ConvE解码器先计算得到模型预测的候选实体的嵌入$\boldsymbol{o}_{t}$:
\begin{equation}
  \boldsymbol{o}_{t}=f\left(vec\left(f\left(\left[\boldsymbol{T}_{cls};\boldsymbol{r}\right]\ast \omega\right)\right)\mathbf{W}\right)
\end{equation}
其中$\ast$代表卷积操作。随后对于任意一个候选实体$e_t$,模型将$\boldsymbol{o}_{t}$与$e_t$的嵌入$\boldsymbol{e}_{t}$进行点积后并经过sigmoid激活函数后得到$e_t$正确的概率:
\begin{equation}
  p_{e_t} = \sigma(\boldsymbol{o}_{t}\cdot \boldsymbol{e}_{t}^T)
\end{equation}
其中$\sigma(\cdot)$为sigmoid激活函数。

获得所有候选实体的得分后,模型采用交叉熵损失函数计算任务损失:
\begin{equation}
  L = -\frac{1}{N}\sum\limits_{i}t_ilog(p_i)+(1-t_i)log(1-p_i)
\end{equation}
$t_i$为第i个候选实体组成的三元组是否正确的标签,$p_i$是模型预测的第i个候选实体组成的三元组是否正确的概率。


\section{本章小结}

本章对NATLP模型的整体架构和实现细节进行了详细介绍。首先对当前基于图神经网络的方法存在的部分问题以及Transformer在知识图谱嵌入领域中应用的限制进行了介绍;随后给出了模型中涉及到的数学符号的详细定义;之后介绍了模型的整体架构组成;最后,对NATLP中的关键设计细节进行了具体的说明,包括(1)关系特定的邻居实体信息构造,利用参数生成网络生成关系特定的网络参数,学习关系对于实体信息传递的作用,并利用全局关系嵌入捕捉不同关系之间的共性。(2)邻域感知Transformer模块,通过最短距离编码和度数编码在自注意力机制计算时融合图结构信息,捕捉邻居消息之间的互相依赖,更好地适应知识图谱的图结构形式。(3)基于卷积神经网络的解码器,利用基于卷积神经网络的方法ConvE进行解码,并计算任务的交叉熵损失。

\chapter{NALTP模型实验与验证}

为了验证本课题提出的NATLP算法的有效性,本文进行了实验验证。本章首先对NALTP的实验方案设计进行了介绍,包括实验采用的数据集、实验环境、评估策略;用来对比的基线算法以及模型在两个数据集上的超参数设置,并对实验结果进行了详细的分析,包括整体性能分析、关键设计分析。

\section{实验方案设计}

\subsection{实验数据集}
为了对NATLP模型进行评估,本文在两个标准基准数据集FB15k-237\upcite{FB15k-237}和WN18RR\upcite{ConvE}上进行了实验。FB15k-237是开源知识图谱Freebase\upcite{freebase}的子集,存储了有关电影、演员、奖项等等现实世界的常识信息的。WN18RR则是开源知识图谱WordNet\upcite{wordnet}的子集,包含了英文单词中的语义信息,例如同义、反义、单词概念的上下层等多种单词语义关系。为了避免测试集出现逆关系泄露的问题,两个数据集中所有的逆关系都已经被去除。两个数据集的统计数据如表\ref{dataset_statistics}所示,其中训练集用于模型参数训练,验证集用于模型超参数调优,测试集用于模型性能评估。值得注意的是,这两个数据集中的关系数量存在差异,FB15k-237包含了237种Freebase中的不同关系,而WN18RR的关系种类数量为11,总的来说,相比于FB15k-237,WN18RR数据集更加的稀疏。

\begin{table}[htbp]
  \renewcommand\arraystretch{1.5}
  \caption{数据集统计信息}
  \centering
  \begin{tabular}{*{3}{c}}
    \toprule
    数据集 & FB15k-237 & WN18RR\\
    \midrule
    \#实体数量  & 10541 & 40943 \\
    \#关系数量 & 237 & 11\\
    \#训练集数量 & 272115 &86835\\
    \#验证集数量 &17535 &3034\\
    \#测试集数量 &20466 &3134\\
    \#实体平均度数 &42.7 &4.5\\
    \bottomrule
  \end{tabular}
  \label{dataset_statistics}
\end{table}

\subsection{实验评估策略}

知识图谱的链路预测任务被定义为实体排序预测任务。测试集中的每个三元组将在两种不同的场景中进行链路预测评估:给定头实体和关系下的尾实体预测$(s,r,?)$,以及给定尾实体和关系下的头实体预测$(?,r,o)$。在实践中,头实体预测以$(o,r^-1,?)$的形式执行。预测时,待预测的头实体或者尾实体将被每个候选实体替换,并计算每个候选三元组的得分,随后所有的候选三元组将按照分数降序进行排序,以获得基本事实三元组的准确排名,并根据排名来对评估指标进行计算。

在知识图谱补全任务中,平均排名(Mean Rank,MR)、倒数平均排名(Mean Reciprocal Rank,MRR)和前N名百分比(Hits@n)三种评估指标将用来评估模型的性能。假设测试集中待预测三元组的数量为$K$,$rank_i$为第$i$个三元组的正确实体在所有候选实体中的排序位置,则平均排序MR的计算方式为:
\begin{equation}
    MR = \frac{\sum_i rank_i}{K}
\end{equation}
MR代表了所有正确实体的平均排序位置,MR越小,说明正确实体的得分越高,排名越靠前,模型的性能更好。但是MR存在的最大问题则是它对于不同排序位置的预测效果投入的关注是一样的,例如,假设有两个三元组,其中一个三元组的正确实体的排名从105上升到了100,另一个三元组的正确实体的排名从5上升到了1,他们对于MR指标的贡献是一致的,但是一般来说,在实验中我们认为从5到1的提升是更有意义的,平均排名指标MR则忽略了这一点。

平均倒数排名MRR对于这个问题进行了改进,MRR计算的时候正确实体排名的倒数的平均值,而不是排名的平均值,具体计算公式为:
\begin{equation}
    MRR = \frac{1}{K} \sum_i \frac{1}{rank_i}
\end{equation}
通过这样的计算方式,排名越靠前的项会获得更大的权重占比,对于整体性能指标的贡献越大。和MR不同,MRR越高,代表模型的性能越好。

前N名百分比Hits@n指的则是所有测试三元组中正确实体的排序处于前n名的比例,计算方式为:
\begin{equation}
    Hits@n = \frac{\sum_i \left[1 \ if \ rank_i \leq n \ or \ 0\right] }{K}
\end{equation}
指标越高,代表模型的性能越好。在知识图谱补全任务中,常用的前N名百分比指标包括Hits@1,Hits@3和Hits@10。

此外,注意到对于一个待预测的三元组$(s,r,?)$,正确实体$o$的选项可能不止一个,例如(姚明,出生于,上海)和(姚明,出生于,中国)都是正确的事实。在计算评价指标时,其余的正确实体可能会导致当前测试三元组中的正确实体排名下降,对模型评估造成影响,因此本文和大多数基线模型类似,计算排序时除了基本事实三元组之外的所有正确三元组都被排除在排名之外。

\subsection{实验环境}

本文采用的实验环境中的服务器硬件和软件配置如表\ref{environment}所示。硬件配置方面,实验所采用的服务器的处理器规格为Intel(R) Core(TM) i9-10920X CPU @ 3.50GHz,内存大小为128G,使用的显卡型号为GeForce RTX 4090,对应的显存大小为24G;软件配置方面,使用的操作系统为Ubuntu 5.4.0-74,采用的编程语言为Python,版本为3.10.9,使用的深度学习框架及其对应的版本为Pytorch 1.13.1。

\begin{table}[htbp]
    \renewcommand\arraystretch{1.5}
    \caption{采用的实验环境}
    \centering
    \begin{tabular}{*{2}{c}}
      \toprule
      配置项目 & 版本/内容\\
      \midrule
      处理器  & Intel(R) Core(TM) i9-10920X CPU @ 3.50GHz \\
      内存 & 128G\\
      显卡 & GeForce RTX 4090\\
      显存 & 24G\\
      操作系统 & Ubuntu 5.4.0-74\\
      编程语言 & Python 3.10.9\\
      深度学习框架 & Pytorch 1.13.1\\
      \bottomrule
    \end{tabular}
    \label{environment}
\end{table}


\subsection{对比算法}

为了验证本文提出的NATLP方法在知识图谱补全任务上的有效性,实验部分本文选取了一些最具代表性的以及最先进的知识图谱补全方法作为基线模型与NATLP模型进行了对比实验,主要包含以下几类方法:

(1)基于翻译的知识图谱嵌入模型,包括TransE\upcite{TransE},RotatE\upcite{RotatE}。

(2)基于张量分解的知识图谱嵌入模型,包括DistMult\upcite{DistMult},ComplEx\upcite{ComplEx}。
 
(3)基于卷积神经网络的模型,包括ConvE\upcite{ConvE},ConvR\upcite{ConvR}。

(4)基于图神经网络的知识图谱嵌入模型,包括R-GCN\upcite{R-GCN},CompGCN\upcite{CompGCN},KBGAT\upcite{KBGAT},HKGN\upcite{HKGN},SE-GNN\upcite{SE-GNN}以及MRGAT\upcite{MRGAT}。

\subsection{超参数设置}

NATLP模型在实验中采用的超参数是通过网格搜索在验证集上进行评估后得出的,主要涉及到的超参数有实体和关系嵌入的维度、关系特定的邻居实体信息构造中使用的卷积核的大小、卷积核的数量、不同神经网络层的Dropout概率、训练批次的大小、训练中的总迭代次数、训练中的最大学习率以及标签的平滑比例等。

实验中采用了Adamax\upcite{Adamax}优化器结合动态学习率调整策略进行模型的训练。在总迭代次数的前10\%内,模型的学习率将从0线性提升到最高,并在剩余迭代次数内线性下降到0。为了防止模型出现过度自信的现象,训练过程中以0.1的比率进行了标签平滑。NATLP模型在两个数据集上的具体的超参数设置见表\ref{NALTP_hyperparameter}所示。

\begin{table}[htbp]
  \renewcommand\arraystretch{1.5}
  \caption{NATLP模型超参数设置}
  \centering
  \begin{tabular}{*{3}{c}}
    \toprule
    超参数 & FB15k-237 & WN18RR\\
    \midrule
    实体和关系嵌入大小  & 320 & 320 \\
    卷积核大小 & 3$\times$3 & 3$\times$3\\
    卷积核数量 & 32 & 32\\
    Transformer网络层数& 4 & 4\\
    嵌入层Dropout概率 & 0.2 & 0.2\\
    Transformer中的Dropout概率 & 0.6 & 0.6\\
    训练批次大小 & 512 & 512\\
    总迭代次数& 300 & 500 \\
    最大学习率 & 0.001 & 0.02\\
    标签平滑比例 & 0.1 & 0.1\\
    \bottomrule
  \end{tabular}
  \label{NALTP_hyperparameter}
\end{table}

\section{实验结果与分析}

\subsection{整体实验结果分析}

不同模型在WN18RR数据集和FB15k-237数据集上的链路预测实验结果如表\ref{NATLP_result_tab}所示。文献\cite{49}在充分大的超参数搜索空间下采用统一的训练框架进行了大量的实验,探究了不同训练策略、模型架构和超参数搜索方法对于知识图谱嵌入模型性能的影响,发现如果选择合适的训练方法,许多浅层的知识图谱嵌入模型能够获得比原始文献中优秀得多的性能。因此TransE,RotatE,DistMult,ComplEx和ConvE模型在两个数据集上的实验结果直接选取文献\cite{49}中经过大量调参后的最优结果。另外,文献\cite{50}指出KBGAT中存在评估策略不当以及测试集数据泄露的问题,因此KBGAT的实验结果选取自文献\cite{50}中经过修正后的结果。

\begin{table}[htbp]
  \begin{center}
      \caption{NATLP实验结果}
      \renewcommand\arraystretch{1.5}
      \setlength{\tabcolsep}{5pt}
      \begin{tabular}{*{11}{c}}
          \toprule
          \multirow[vpos]{2}{*}[-0.8ex]{模型} & \multicolumn{5}{c}{WN18RR} & \multicolumn{5}{c}{FB15k-237}\\
          \cmidrule(lr){2-6}\cmidrule(lr){7-11}
          &MRR & MR & Hits@1 &Hits@3 & Hits@10&MRR & MR & Hits@1 &Hits@3 & Hits@10\\
          \midrule
          TransE&0.228&-&0.053&0.368&0.520&0.313&-&0.221&0.347&0.497\\
          RotatE&0.478&-&0.439&0.494&0.553&0.333&	-&0.240&0.368&0.522\\
          \cmidrule{1-11}
          DistMult&0.452&-&0.413&0.466	&0.530	&0.343	&-	&0.250	&0.378	&0.531\\
          ComplEx&0.475&-&0.438	&0.490	&0.547	&0.348	&-	&0.253	&0.384	&0.536\\
          \cmidrule{1-11}
          ConvE&0.442&-&0.411&0.451	&0.504&	0.339	&-&	0.248&	0.369	&0.521\\
          ConvR&0.475&-&0.443&	0.489&	0.537&	0.350&	-&	0.261&	0.385&	0.528\\
          \cmidrule{1-11}
          R-GCN & -&-&-&-&-&0.248&-&0.153&0.258&0.414\\
          CompGCN&0.479&3533&0.443	&0.494	&   0.546&	0.355	&197	&0.264	&0.390&	0.535\\
          KBGAT&0.412&\textbf{1921}&-&	-&	0.554	&0.157	&270	&-&	-&	0.331\\
          SE-GNN&0.484&3211&0.446&	\underline{0.509}&	\underline{0.572}	&\underline{0.365}& \textbf{157}	&\underline{0.271}	&\underline{0.399}	&\underline{0.549}\\
          MRGAT&0.481&-&0.443&0.501&0.568&0.358	&-&0.266&0.386&0.542\\
          HKGN&\underline{0.487}&\underline{2468}&\underline{0.448}&0.505&0.561&\underline{0.365}&194&\underline{0.271}&0.397&0.544\\
          \cmidrule{1-11}
          \textbf{NATLP}&\textbf{0.505}&2687&\textbf{0.465}&\textbf{0.519}&\textbf{0.576}&\textbf{0.374}&\underline{181}&\textbf{0.281}&\textbf{0.411}&\textbf{0.560}\\
          % TKGE-PN&\textbf{0.510}&2540&\textbf{0.467}	&\textbf{0.522}	&\underline{0.590}	&\textbf{0.379}	&160	&\underline{0.288}	&\textbf{0.414}	&\textbf{0.562}\\
          \bottomrule
      \end{tabular}
      \label{NATLP_result_tab}
  \end{center}
\end{table}

表中加粗项为每项指标的最高值,下划线项为每项指标的次高值。从表\ref{NATLP_result_tab}中可以观察到,在两个基准数据集的绝大多数指标上,NATLP都取得了最优的效果,相对于次优结果,在FB15k-237数据集上,NATLP在MRR、Hits@1、Hits@3和Hits@10四个指标上分别取得了3.6\%,3.7\%, 1.9\%, 0.6\%的性能提升;在WN18RR数据集上,NATLP在MRR、Hits@1、Hits@3和Hits@10四个指标上分别取得了2.4\%,3.6\%, 3.0\%和2.0\%的性能提升,这表明本文提出的NALTP方法在链路预测任务上有着很好的表现,证明了NATLP模型的有效性。同样使用了注意力机制,相比于HKGN以及MRGAT等基于图神经网络的方法,NATLP有着可观的性能提升,证明了Transformer网络强大的表达能力以及在知识图谱嵌入领域的应用潜力。

\subsection{模型关键设计分析}

为了验证NATLP模型中关键设计:关系特定的邻居实体信息构造以及邻域感知Transformer模块的有效性,本文进行了多方面的对比实验来验证关键设计的效果。

首先,为了验证关系特定的邻居实体信息构造中参数生成网络以及全局关系嵌入对于链路预测任务的作用,本文在两个数据集上分别以三种设置进行了消融实验,分别是(1)原始NATLP模型,未对模型进行任何改动;(2)消融了参数生成网络的NATLP模型,同一实体在不同关系下进行信息构造中共享完全相同的网络参数;(3)消融了全局关系嵌入的NATLP模型,参数生成网络仅依赖于关系嵌入进行参数生成。实验结果如表\ref{NATLP_ablation1}所示。

\begin{table}[htbp]
  \begin{center}
      \caption{关系特定的邻居实体信息构造消融实验}
      \setlength{\tabcolsep}{8pt}
      \renewcommand\arraystretch{1.5}
      % \renewcommand{\arraystretch}{1}
      \begin{tabular}{*{7}{c}}
          \toprule
          数据集 & 模型设置 & MRR&MR&Hits@1&Hits@3&	Hits@10\\
          \midrule
          \multirow{3}{*}{WN18RR}&NATLP&0.505&2504&0.466&0.519&0.578\\
          &消融参数生成网络&0.496&2399&0.456&0.512&0.575\\
          &消融全局关系嵌入&0.503&2519&0.461&0.518&0.578\\
          \cmidrule{1-7}
          \multirow{3}{*}{FB15k-237}&NATLP&0.376&161&0.284&0.411&0.561\\
          &消融参数生成网络&0.367&145&0.274&0.404&0.555\\
          &消融全局关系嵌入&0.373&175&0.279&0.411&0.558\\
          \bottomrule
      \end{tabular}
      \label{NATLP_ablation1}
  \end{center}
\end{table}

从表\ref{NATLP_ablation1}中实验结果可以观察到,在对参数生成网络进行消融后,模型的在两个数据集上的链路预测性能都有了明显的下降,这证明了相比于采用统一的网络参数进行消息构造,关系特定的网络参数能够更好地捕捉实体中关系相关的特定特征,挖掘相关的语义信息,提高任务性能;注意到消融了参数生成网络之后,相比于WN18RR数据集上的表现,NATLP模型在FB15k-237数据集上的性能下降更为严重,本文认为这样的差异主要是由于两个数据集上的关系种类差异所导致的,FB15k-237数据集中的关系种类为237种,远远超过WN18RR数据集上的11种,因此采用统一的网络参数时模型在FB15k-237数据集上损失的信息更多。此外,去除全局关系嵌入后模型在两个数据集上的性能也有所下降,证明了捕捉关系之间共同特征的有效性。

此外,为了验证邻域感知Transformer模块中基于实体节点间最短距离的偏置项以及基于节点度数的偏置项的有效性,本文在两个数据集上以四种设置进行了NATLP模型的消融实验,分别是(1)原始NATLP模型,未对模型进行任何改动;(2)消融了基于实体节点间最短距离的偏置项的NATLP模型;(3)消融了基于节点度数的偏置项的NATLP模型;(4)同时对两个偏置项进行了消融的NATLP模型。具体实验结果如表\ref{NATLP_ablation2}所示。

\begin{table}[htbp]
  \begin{center}
      \caption{邻域感知Transformer模块消融实验}
      \setlength{\tabcolsep}{8pt}
      \renewcommand\arraystretch{1.5}
      % \renewcommand{\arraystretch}{1}
      \begin{tabular}{*{7}{c}}
          \toprule
          数据集 & 模型设置 & MRR&MR&Hits@1&Hits@3&	Hits@10\\
          \midrule
          \multirow{4}{*}{WN18RR}&NATLP&0.505&2504&0.466&0.519&0.578\\
          &消融节点最短距离的偏置项&0.500&2603&0.457&0.515&0.573\\
          &消融节点度数偏置项&0.501&2570&0.459&0.517&0.578\\
          &消融两个偏置项&0.497&2856&0.454&0.515&0.570\\
          \cmidrule{1-7}
          \multirow{4}{*}{FB15k-237}&NATLP&0.376&161&0.284&0.411&0.561\\
          &消融节点最短距离的偏置项&0.373&180&0.280&0.409&0.557\\
          &消融节点度数偏置项&0.372&184&0.281&0.409&0.554\\
          &消融两个偏置项&0.369&179&0.277&0.409&0.550\\
          \bottomrule
      \end{tabular}
      \label{NATLP_ablation2}
  \end{center}
\end{table}

通过表中结果可以发现,无论是消融了节点间最短距离的偏置项还是节点度数的偏置项,模型在两个数据集上的性能都会出现下降,证明了两种结构信息对于链路预测任务的重要性。


\section{本章小结}

本章对于基于领域感知的Transformer模型NALTP的实验部分进行了介绍,首先对本文中实验的基本方案设计进行了说明,包括实验采用的数据集,模型的评估策略以及实验的硬件和软件环境,并介绍了用来对比NATLP的基线模型。随后论文对NALTP模型的整体实验结果进行了介绍,证明了模型的有效性;最后通过消融实验对NATLP模型的中的关键设计:关系特定的邻居实体信息构造以及邻域感知Transformer模块进行了具体的探究和分析。

\chapter{结合图路径和局部邻域的Transformer模型}

本章主要对结合图路径和局部邻域的Transformer模型TKGE-PN的总体设计和模块的具体实现进行了介绍。主要包括对基于图神经网络的方法以及NATLP模型存在问题的分析、模型的总体框架设计、基于有偏随机游走的图路径采样算法的设计、Path-Transformer路径编码模块以及Neighbor-Transformer局部领域编码模块的具体设计。

\section{现有问题分析}
NATLP模型对传统的Transformer模型的自注意力机制的计算方式进行改进,捕获了中心实体局部邻域内的结构信息,将Transformer模型应用到了知识图谱补全中,解决了先前基于图神经网络的方法模型表达能力差、对邻居实体之间相互依赖学习不足的问题。

但是,NATLP模型依然存在缺陷,由于依然基于中心实体的局部邻域进行推理,NATLP并没有解决基于图神经网络的知识图谱嵌入方法对于于长距离依赖学习不足的缺点。在知识图谱中,长距离的含义是实体间通过边连接需要经过多个中间节点。长距离依赖反映出实体之间较为间接的联系,这种联系在理解实体间复杂关系的层面上是非常重要的。例如,某一个历史人物与某一个现代组织之间可能存在长距离依赖,尽管他们之间直接的联系很少,但通过一系列历史事件和影响,可以构建出两者之间的联系,因此捕捉长距离依赖有益于推理和查询知识图谱的高阶模式。

基于图神经网络的方法主要是通过利用中心实体的局部邻域中蕴含的信息来完成知识图谱的补全。每层图神经网络只能学习到中心实体的一跳邻居的信息,这导致模型能够很好的捕捉知识图谱中的短距离依赖,而对于实体和实体之间的长距离依赖的学习不够充分。虽然其可以通过堆叠多层图神经网络让中心实体感知到距离更远的其他实体,但这样的方式只在层数较低(例如一到两层)时有效,之后随着层数的进一步增加,模型的性能反而会出现快速的下降,这主要是由过平滑问题(over-smoothing)导致的,即因为感受野的重叠而导致不同节点的表示过于相似无法进行区分。文献\cite{over-smoothing}对这种现象进行了系统和定量的研究,通过引入定量指标,发现导致过平滑的关键因素是节点接收到的有效信息与噪声比例过低,结果如图\ref{information-noise}所示,该图来自于文献\cite{over-smoothing}原文。在每层神经网络中,节点都会聚合其所有邻居节点的信息并传递给下一层,这导致模型引入了大量的噪声信息。因此基于图神经网络的方法一般只能捕捉单个实体附近1-2跳内的局部信息,而缺乏利用长距离乃至全局信息的能力。
\begin{figure}[htb]
  \centerline{\includegraphics[width=0.7\textwidth]{pic/information-noise.png}}
  \caption{网络层数对信噪比的影响}
  \label{information-noise}
\end{figure}

为了解决以上问题,本文提出了一种基于Transformer的结合图路径和局部邻域的知识图谱嵌入方法(A Transformer-based Knowledge Graph Embedding Model Combining Graph Paths and Local Neighborhood,TKGE-PN)。基于图神经网络的方法的成功证明了实体的局部邻域蕴含了丰富的信息,但知识图谱的结构信息除了图神经网络使用的局部邻域之外还有多种表达形式,例如图路径以及子图。在知识图谱中,图路径被定义为图谱中的实体-关系链,由不同三元组收尾相连组成,例如(Yao Ming, Born In, Shanghai, City Of, China)。相对于局部邻域,图路径能够帮助模型更好地捕获实体和实体之间长距离的依赖,如图\ref{long-term-dependency}所示。结合图路径和邻域信息,模型能够更好地学习长短距离依赖的同时避免过度平滑问题的出现。同样的,和NATLP类似,和GNN浅层的神经网络结构相比,Transformer的自注意力机制能够给模型带来更强大的表达能力。本文提出的TKGE-PN以中心实体作为起点,采用有偏随机游走算法对图路径进行采样,并通过基于Transformer的图路径编码模块和邻域信息编码模块对图谱中的长距离和短距离结构信息进行编码。此外本文为图路径编码模块设计了一个掩蔽实体关系预测任务,以确保模型能够充分学习图路径之间的长距离依赖。

\begin{figure}[htb]
    \centerline{\includegraphics[width=0.5\textwidth]{pic/long-term-dependency.pdf}}
    \caption{知识图谱中的短距离距离信息和长距离信息}
    \label{long-term-dependency}
  \end{figure}

\section{TKGE-PN模型设计}

\subsection{符号定义}
为了方便说明论文提出的TKGE-PN模型的实现细节,本节对TKGE-PN模型中的关键概念和相关的数学符号进行了定义,具体内容参见表\ref{definition_TKGE-PN}。

\setlength{\tabcolsep}{20pt}

\renewcommand\arraystretch{1.2}
\begin{longtable}[htbp]{cc}
  % 首页表头
  \caption{TKGE-PN模型中的符号定义}
  \label{definition_TKGE-PN}\\
  \toprule
  符号  & 说明\\
  \midrule
  \endfirsthead
  % 续页表头
  \caption{TKGE-PN模型中的符号定义}\\
  \toprule
  符号  & 说明 \\
  \midrule
  \endhead
  % 首页表尾
  \hline
  % \multicolumn{2}{r}{\small 续下页}
  \endfoot
  % 续页表尾
  \bottomrule
  \endlastfoot
  
  $\mathcal{G}$   &   知识图谱      \\
  $\mathcal{E}, \mathcal{R}, \mathcal{T}$   &   实体集合、关系集合、边集合      \\
  $\mathcal{G}^\prime$  &  拓展后的知识图谱      \\
  $\mathcal{R}^{\prime}$   &   拓展后的关系集合      \\
  $\mathcal{T}^{-1}$   &   逆关系边集合      \\
  $\mathcal{T}^{\prime}$   &   拓展后的边集合      \\
  $(s,r,?)$  &   待遇测的三元组      \\
  $s$   &   头实体即中心实体      \\
  $o$   &   尾实体即目标实体      \\
  $e$   &   实体      \\
  $r$   &   关系      \\
  $r^{-1}$   &   关系$r$的逆关系      \\
  $\boldsymbol{s},\boldsymbol{o}$ & 头实体嵌入和尾实体嵌入\\
  $\boldsymbol{e},\boldsymbol{r}$ & 实体嵌入和关系嵌入\\
  $d$ &嵌入维度\\
  $\phi_{chk}$ & 棋盘式特征重组\\
  $\circledast$ & 循环卷积操作\\
  $f(\cdot )$ & ReLU激活函数\\
  $vec(\cdot)$ & 二维张量转化为一维向量\\
  $\omega_r$ & 特定于关系$r$的卷积层参数\\
  $\mathbf{W}_r$ &特定于关系$r$的全连接层参数\\
  $\boldsymbol{e}_{cls}$ & 特殊嵌入Class Token\\
  $\mathbf{TE}$ & 类型嵌入\\
  $a_{ij}$ & 第i个输入和第j个输入之间的注意力得分\\
  $dis(e_i,e_j)$ & 实体$e_i$与实体$e_j$之间最短路径的距离\\
  $deg(e)$ & 实体$e$的节点度数\\
  $\boldsymbol{o}_t$ & 模型预测的候选实体的嵌入\\
  $\sigma $ & sigmoid激活函数\\
  $p$ & 三元组正确概率\\
  $L$ & 模型损失\\
  $t_i$ & 第i个三元组的标签\\

\end{longtable}

和NATLP中的处理方式类似,为了确保信息的双向流动,TKGE-PN会对知识图谱进行拓展,为知识图谱中的每个事实三元组$(s,r,o)$添加对应的逆关系和逆三元组:

\begin{gather}
    \mathcal{R}^{\prime}=\mathcal{R}\cup\{ r^{-1} | r\in \mathcal{R}\}\\
    \mathcal{T}^{-1}= \{ (o,r^{-1},s)| (s,r,o)\in \mathcal{T}\}\\
    \mathcal{T}^{\prime} = \mathcal{T}\cup\mathcal{T}^{-1}\\
    \mathcal{G}^\prime = (\mathcal{E}, \mathcal{R}^\prime, \mathcal{T}^\prime)
\end{gather}

\subsection{模型总体结构}

本节主要对提出的基于Transformer的结合图路径和局部邻域的知识图谱嵌入方法TKGE-PN的总体结构进行介绍。和NATLP不同的是,TKGE-PN并没有采用编码器-解码器架构,而是利用Transformer自身的强大表达能力直接对目标实体的嵌入进行预测,这样的方式的优点是模型的性能不会受限于用作解码器的基于图神经网络的知识图谱嵌入方法的限制。

TKGE-PN模型主要由三个部分组成,首先是基于有偏随机游走的图路径采样算法,以中心实体为起点在图谱中采样多条图路径;随后Path-Transformer路径编码模块负责学习图路径中蕴含的长距离的语义信息并转换为向量表示;Neighbor-Transformer局部邻域编码模块接收来自Path-Transformer的输入,整合待遇测事实三元组以及多条图路径所构成上下文邻域信息并预测三元组得分。模型架构如图\ref{TKGE-PN_architecture}所示。

\begin{figure}[htb]
  \centerline{\includegraphics[width=1\textwidth]{pic/TKGE-PN_architecture.pdf}}
  \caption{TKGE-PN模型整体架构}
  \label{TKGE-PN_architecture}
\end{figure}


\chapter{TKGE-PN模型实验与验证}
为了验证本文提出的TKGE-PN算法能够通过结合图路径和局部邻域信息实现对知识图谱中的长短距离依赖的综合学习,提高链路预测任务的性能表现,本文在两个基准数据集上对TKGE-PN进行了实验验证。TKGE-PN模型实验时采用的数据集、实验环境和评估策略和NATLP模型实验时完全一致,因此本章不再另行介绍。本章首先说明了用来对比的基线算法以及模型在两个数据集上的超参数设置,随后介绍了模型的整体实验结果并进行了深入的探究,最后通过消融实验对模型的关键设计进行了分析和讨论。

\section{实验方案设计}

\subsection{对比算法}

为了验证本文提出的TKGE-PN模型在链路预测任务上的有效性,在实验部分,本文将其与一些最具代表性的以及最先进的知识图谱嵌入模型进行了比较,主要包含以下几类:

(1)传统的知识图谱嵌入模型,时间和空间复杂度较低,包括基于翻译的模型TransE\upcite{TransE},RotatE\upcite{RotatE},基于张量分解的模型DistMult\upcite{DistMult},ComplEx\upcite{ComplEx},以及基于卷积神经网络的模型ConvE\upcite{ConvE}和ConvR\upcite{ConvR}。

(2)基于图神经网络的嵌入模型,主要学习图谱局部邻域结构进行知识图谱嵌入,包括R-GCN\upcite{R-GCN},CompGCN\upcite{CompGCN},KBGAT\upcite{KBGAT},HKGN\upcite{HKGN},SE-GNN\upcite{SE-GNN}以及MRGAT\upcite{MRGAT}。

(3)基于图路径的嵌入模型,利用知识图谱的图路径结构进行嵌入工作,包括RSN\upcite{RSN}和Interstellar\upcite{Interstellar}。 

(4)基于Transformer的嵌入模型,主要利用Transformer的表达能力进行链路预测,包括KG-BERT\upcite{KG-BERT},HittER\upcite{HittER},StAR\upcite{StAR}和Relphormer\upcite{Relphormer},以及本文先前提出的NALTP模型。

\subsection{模型超参数设置}

实验采用的超参数是通过网格搜索在验证集上进行评估后得出的。实验中,Path-Transformer图路径编码模块以及Neighbor-Transformer局部邻域编码模块分别由3层和4层Transformer编码层组成,多头注意力机制的头数为8,采用的实体和关系嵌入维度为320。在路径采样算法的设置上,两个数据集上路径采样的深度偏差权重$\alpha = 0.7$,度数偏差权重$\beta=0.1$,WN18RR数据集上采样路径的最大长度$T=4$,FB15k-237数据集上$T=3$。在掩蔽实体关系预测任务中,每个实体和关系有30\%的概率被特殊的掩蔽占位嵌入$e_{mask}$替换,20\%被其余随机实体和关系替换,50\%的概率维持不变。考虑到数据集中存在极小一部分节点度数极高的实体以及FB15k-237和WN18RR数据集在节点平均度数上的差异,实验对每个实体采样得到的图路径的最大数量进行了限制,在FB15k-237和WN18RR数据集上分别为50和12,这样的设置能够保证采样的图路径可以完全覆盖数据集中80\%以上实体的一阶邻域。

实验中采用了Adamax\upcite{Adamax}优化器结合动态学习率调整策略进行模型的训练。在总迭代次数的前10\%内,模型的学习率将从0线性提升到最高,并在剩余迭代次数内线性下降到0。在训练过程中,除了嵌入层以外的神经网络层的随机失活概率为0.65,嵌入层的随机失活概率为0.2。模型训练时在FB15K-237数据集上的总迭代次数为300,训练批次的大小为512,最大学习率为0.0011;而对于WN18RR数据集,总迭代次数为500,批处理大小为512,最大学习率为0.002。为了防止模型出现过度自信的现象,训练过程中以0.1的比率进行了标签平滑。模型使用的具体超参数参见表\ref{TKGE-PN_hyperparameter}。

\begin{table}[htbp]
    \renewcommand\arraystretch{1.5}
    \caption{TKGE-PN模型超参数设置}
    \centering
    \begin{tabular}{*{3}{c}}
      \toprule
      超参数 & FB15k-237 & WN18RR\\
      \midrule
      实体和关系嵌入大小  & 320 & 320 \\
      Path-Transformer网络层数& 3 & 3\\
      Neighbor-Transformer网络层数& 4 & 4\\
      嵌入层Dropout概率 & 0.2 & 0.2\\
      Transformer中的Dropout概率 & 0.65 & 0.65\\
      采样路径的最大长度&3 &4\\
      深度偏差权重& 0.7 & 0.7\\
      度数偏差权重& 0.1 & 0.1\\
      路径采样最大数量 &50&12\\
      训练批次大小 & 512 & 512\\
      总迭代次数& 300 & 500 \\
      最大学习率 & 0.001 & 0.02\\
      标签平滑比例 & 0.1 & 0.1\\
      \bottomrule
    \end{tabular}
    \label{TKGE-PN_hyperparameter}
  \end{table}

\section{实验结果与分析}

\subsection{整体实验结果分析}

不同模型在WN18RR数据集和FB15k-237数据集上的链路预测实验结果如表所示。和NATLP类似,TransE、RotatE、DistMult、ComplEx和ConvE在两个数据集上的实验结果直接选取文献\cite{49}中经过大量调参的最优结果。由于存在评估策略不当以及测试集数据泄露的问题,KBGAT的实验结果选取文献\cite{50}中经过修正后的结果。

\begin{table}[htbp]
    \begin{center}
        \caption{TKGE-PN实验结果}
        \renewcommand\arraystretch{1.5}
        \setlength{\tabcolsep}{5pt}
        \begin{tabular}{*{11}{c}}
            \toprule
            \multirow[vpos]{2}{*}[-0.8ex]{模型} & \multicolumn{5}{c}{WN18RR} & \multicolumn{5}{c}{FB15k-237}\\
            \cmidrule(lr){2-6}\cmidrule(lr){7-11}
            &MRR & MR & Hits@1 &Hits@3 & Hits@10&MRR & MR & Hits@1 &Hits@3 & Hits@10\\
            \midrule
            TransE&0.228&-&0.053&0.368&0.520&0.313&-&0.221&0.347&0.497\\
            RotatE&0.478&-&0.439&0.494&0.553&0.333&	-&0.240&0.368&0.522\\
            DistMult&0.452&-&0.413&0.466	&0.530	&0.343	&-	&0.250	&0.378	&0.531\\
            ComplEx&0.475&-&0.438	&0.490	&0.547	&0.348	&-	&0.253	&0.384	&0.536\\
            ConvE&0.442&-&0.411&0.451	&0.504&	0.339	&-&	0.248&	0.369	&0.521\\
            ConvR&0.475&-&0.443&	0.489&	0.537&	0.350&	-&	0.261&	0.385&	0.528\\
            \cmidrule{1-11}
            R-GCN & -&-&-&-&-&0.248&-&0.153&0.258&0.414\\
            CompGCN&0.479&3533&0.443	&0.494	&   0.546&	0.355	&197	&0.264	&0.390&	0.535\\
            KBGAT&0.412&1921&-&	-&	0.554	&0.157	&270	&-&	-&	0.331\\
            SE-GNN&0.484&3211&0.446&	0.509&	0.572	&0.365& 157	&0.271	&0.399	&0.549\\
            MRGAT&0.481&-&0.443&0.501&0.568&0.358	&-&0.266&0.386&0.542\\
            HKGN&0.487&2468&0.448&0.505&0.561&0.365&194&0.271&0.397&0.544\\
            \cmidrule{1-11}
            RSN&0.400&-&0.380	&-	&0.448&	0.280	&-	&0.202	&-	&0.453\\
            Interstellar&0.480&-&0.438	&-	&0.546	&0.320	&-	&0.233	&-	&0.508\\
            \cmidrule{1-11}
            KG-BERT&0.216&\underline{97}&0.041&	0.302	&0.524	&-	&\underline{153}	&-&	-&	0.420\\
            HittER&0.503&-&0.462&	0.516&	0.584	&0.373	&-	&0.279	&0.409	&0.558\\
            StAR&0.40&\color{red}\textbf{51}&0.243	&0.491	&0.709	&0.296	&\color{red}\textbf{117}&	0.205	&0.322	&0.482\\
            Relphormer&0.495&-&0.448&	-	&\color{red}\textbf{0.591}&	0.371		&-&\color{red}\textbf{0.314}	&-	&0.481\\
            \cmidrule{1-11}
            NATLP&\underline{0.505}&2687&\underline{0.465}&\underline{0.519}&0.576&\underline{0.374}&181&0.281&\underline{0.411}&\underline{0.560}\\
            \color{red}\textbf{TKGE-PN}&\color{red}\textbf{0.510}&2540&\color{red}\textbf{0.467}	&\color{red}\textbf{0.522}	&\underline{0.590}	&\color{red}\textbf{0.379}	&160	&\underline{0.288}	&\color{red}\textbf{0.414}	&\color{red}\textbf{0.562}\\
            \bottomrule
        \end{tabular}
        \label{TKGE-PN_result_tab}
    \end{center}
  \end{table}

  表中加粗项为每项指标的最高值,下划线项为每项指标的次高值。从表\ref{TKGE-PN_result_tab}中可以观察到,(1)在两个基准数据集中的大多数评价指标上,TKGE-PN模型优于其他所有的基准模型,这证明了本文提出的方法的有效性。TKGN-PN和NATLP等基于Transformer的嵌入方法相比于基于图卷积神经网络的嵌入方法如MRGAT取得了较大的性能提升,这证明了Transformer结构在知识图谱嵌入领域的巨大潜力。(2)而同样是基于Transformer的知识图谱嵌入方法,NALTP利用Transformer来学习中心实体的一阶邻域信息,没有利用路径信息而缺乏捕捉图谱中长距离依赖的能力。TKGE-PN相对于NALTP的性能提升表明了图路径结构对于知识图谱嵌入的帮助以及TKGE-PN从图路径结构中挖掘长距离依赖的能力。

  为了进一步研究图路径采样策略对于TKGE-PN链路预测性能的影响,本文探究了在不同图路径采样长度下TKGE-PN在两个数据集上的性能表现,实验结果如表\ref{length_tab}所示。

\begin{table}[htbp]
    \begin{center}
        \caption{验证集上不同路径采样长度下的链路预测结果}
        %\setlength{\tabcolsep}{12pt}
        \renewcommand{\arraystretch}{1.5}
        \begin{tabular}{*{5}{c}}
            \toprule
            采样路径的最大长度& \multicolumn{2}{c}{WN18RR} & \multicolumn{2}{c}{FB15k-237}\\
            \cmidrule(lr){2-3}\cmidrule(lr){4-5}
            $T$&MRR&Hits@10&MRR&Hits@10\\
            \midrule
            1	&0.498&0.571&0.373&0.559\\
            2	&0.503&0.580&0.376&0.560\\
            3	&0.506&0.584&\textbf{0.380}&\textbf{0.563}\\
            4	&\textbf{0.509}&\textbf{0.588}&0.377&0.561\\
            5	&0.507&0.585&0.375&0.557\\
            6	&0.503&0.578&0.372&0.559\\
            7	&0.501&0.576&0.370&0.555\\
            \bottomrule
        \end{tabular}
        \label{length_tab}
    \end{center}
\end{table}
  
可以发现,随着采样长度的增加,模型在两个数据集上的性能总体呈现一个先上升后降低的趋势。一方面,路径长度的增加能够帮助模型学习到长距离的依赖,因此在初期能够有效提升模型嵌入效果;而另一方面,并不是所有的图路径信息都是有意义的,采样长度过高时采样到的路径随机性会增大,距离过长时两个实体之间的依赖也会减弱,引入的噪声信息随之增加,反而对模型造成了干扰,导致性能下降。此外,根据实验结果可以发现,在不同数据集上模型最优的采样长度也不同,相比较之下,WN18RR数据集上的最优长度更大。本文认为这样的差异是由于不同数据集之间的稀疏性差距导致的,WN18RR数据集相对更加稀疏,使得采样的随机性降低,引入的噪声数据减少,因此最优的采样长度相比于FB15k-237数据集更大。

此外,论文还探究了不同路径深度偏差$\alpha$下TKGE-PN模型的性能表现。深度偏差$\alpha$较小时路径采样偏向于围绕中心实体进行,深度偏差$\alpha$较大时则更注重于捕捉远距离的依赖。根据实验结果,论文发现当深度偏差$\alpha\in\left[0.6,0.8\right] $时模型的性能表现最好。本文认为当深度偏差控制在这个区间内时,模型能够比较好的兼顾长短距离的信息。

\subsection{模型关键设计分析}

为了验证TKGE-PN模型中局部邻域、图路径信息以及掩蔽实体关系预测任务对于知识图谱嵌入的作用,论文在两个数据集上分别以四种设置进行了消融实验,分别是未消融任何部分的TKGE-PNM模型、去除了局部邻域编码模块、去除掩蔽实体关系预测任务以及同时去除图路径编码模块和掩蔽实体关系预测任务,实验结果如表\ref{ablation_tab}所示。

\begin{table}[htbp]
    \begin{center}
        \caption{TKEG-PN消融实验结果}
        \setlength{\tabcolsep}{10pt}
        \renewcommand{\arraystretch}{1.5}
        \begin{tabular}{*{7}{c}}
            \toprule
            数据集 & 模型 & MRR&MR&Hits@1&Hits@3&	Hits@10\\
            \midrule
            \multirow{4}{*}{WN18RR}&TKGE-PN&0.509&2610&0.467&0.525&0.588\\
            &Path+MERP&0.487&2752&0.445&0.504&0.567\\
            &Neighbor&0.497&2806&0.459&0.512&0.570\\
            &Neigbor+Path&0.502&2636&0.461&0.521&0.579\\
            \cmidrule{1-7}
            \multirow{4}{*}{FB15k-237}&TKGE-PN&0.380&156&0.289&0.415&0.563\\
            &Path+MERP&0.367&144&0.278&0.406&0.555\\
            &Neighbor&0.373&175&0.279&0.410&0.559\\
            &Neigbor+Path&0.376&169&0.282&0.413&0.562\\
            \bottomrule
        \end{tabular}
        \label{ablation_tab}
    \end{center}
\end{table}

具体来说,去除局部邻域编码模块是通过限制路径采样数量为1来实现,而去除图路径编码模块则是通过限制路径采样长度为1实现,此时路径信息内只包含中心实体的一阶邻居,即一阶局部邻域信息。可以看到,无论是去除局部邻域编码模块,还是去除图路径编码模块,模型的效果都会出现明显的降低。实验证明了知识图谱中的这两种结构信息都能够帮助进行链路预测,仅依赖路径信息,模型无法从局部邻域内的丰富的实体和关系中学习到中心实体的综合性质;仅依赖局部邻域,则无法挖掘到中心实体和其他实体之间的远距离依赖。相对来说,去除局部邻域编码模块带来的性能下降更加的严重,这证明了局部邻域信息的重要性,也符合一般的直觉:距离中心实体更近的实体和关系更能反映中心实体的性质,蕴含的信息也更加容易学习。此外,论文发现在脱离掩蔽实体关系预测任务的情况下添加图路径信息而并不能带来明显的性能提升,这说明了图路径上长距离信息学习的困难性以及掩蔽实体关系预测对提升模型长距离信息学习能力的作用。总体来看,TKGE-PN模型任一部分的缺失都会导致模型的最终结果受到负面影响。

为了进一步验证图路径信息在挖掘知识图谱中长距离依赖的作用,以及长距离依赖信息对于链路预测任务的帮助,本文将WN18RR验证集中的事实三元组按照头实体和尾实体之间的最短距离进行了划分,并在划分后的验证集上评估了消融了路径编码模块的TKGE-PN模型以及原始TKGE-PN模型的链路预测任务性能,实验结果如图\ref{tkge_experiment}所示。

\begin{figure}[htb]
    \centerline{\includegraphics[width=1\textwidth]{pic/TKGE-PN_experiment.pdf}}
    \caption{WN18RR分组实验结果}
    \label{tkge_experiment}
  \end{figure}

从图\ref{tkge_experiment}中实验结果可以发现,随着头实体和尾实体之间最短距离的增加,模型的MRR指标出现了明显的下降,说明推断图谱之间的长距离依赖是非常困难的。 此外,可以注意到的是,相比于消融了图路径编码模块的模型,学习了图路径的TKGE-PN模型在长距离的样本上性能表现更好,特别是在头尾实体间最短距离为2-4的样本上,证明了图路径信息能够有效捕捉实体间的长距离依赖,提高链路预测任务的性能表现。

\section{本章小结}

本章对于结合图路径和局部邻域的Transformer模型TKGE-PN的实验部分进行了介绍,首先介绍了用来对比的基线模型以及模型在两个数据集上的超参数设置。之后论文对TKGE-PN模型的整体实验结果进行了介绍,证明了TKGE-PN模型性能表现优于绝大部分现有的模型。论文随后通过实验进一步探究了路径长度以及路径采样策略对于TKGE-PN性能的影响;最后论文通过实验对TKGE-PN模型中的关键设计进行了具体的探究和分析,包括图路径编码模块、局部邻域编码模块以及掩蔽实体关系预测任务,并设计了实验证明了图路径在长距离依赖信息学习中的重要作用。

% 总结
% !TeX root = ../Template.tex
% 总结
\summary
% 自从2017年被提出,Transformer已经在计算机视觉、自然语言处理等多个领域大放异彩,展现出了巨大的潜力,收到了许多的关注,因此也有不少工作尝试将Transformer网络引入到知识图谱补全领域中。
本课题针对传统知识图谱嵌入方法和基于图神经网络的方法的缺点,研究如何利用Transformer模型来学习知识图谱中的语义和结构信息,提升知识图谱补全任务的性能,提出了两种新型的基于Transformer的知识图谱补全方法:基于邻域感知的Transformer模型NATLP,并在NATLP模型的基础上进一步提出了结合图路径和局部邻域的Transformer模型TKGE-PN。

针对基于图神经网络的知识图谱嵌入方法表达能力不足的问题,NATLP研究利用Transformer网络来完成知识图谱补全。针对以往模型对关系和实体间的交互建模不足的问题,NATLP在模型输入信息构造阶段,基于关系生成特定的网络参数,实现关系特定的邻居信息构造,显示建模了不同关系对于实体传递消息的影响;针对Transformer无法直接感知图结构的问题,NALTP对Transformer的自注意力机制进行了改造,提出了一种融合图结构的自注意力机制,使得模型能够学习到输入消息之间的互相依赖;

NATLP模型解决了基于图神经网络的模型存在的部分问题,但依然存在无法有效学习长距离依赖的缺陷。针对基于图神经网络的方法和NATLP方法中的以上缺陷,本文在NATLP的基础上进一步提出了结合图路径和局部邻域的Transformer模型TKGE-PN。TKGE-PN通过对知识图谱中的局部邻域和图路径两种结构信息的融合,完成了在利用丰富的邻域信息的同时对于图谱中长距离信息的挖掘,提高了知识图谱补全任务的性能。TKGE-PN首先通过基于有偏随机游走的采样算法对图路径进行采样,随后通过基于Transformer的图路径编码模块学习其中的长距离依赖,并通过掩蔽实体关系任务实现长短距离信息的平衡;最后通过局部邻域编码模块实现了图路径和局部邻域结构信息的综合应用。

实验结果表明,在两个标准数据集WN18RR和FB15k-237上,NATLP和TKGE-PN的链路预测任务的性能表现超越绝大多数现有的嵌入模型,证明了本文提出的两个模型以及其中的关键设计的有效性。

% 未来,我们计划进一步探索对于知识图谱中结构信息的利用,包括更有效率的利用方式和更加丰富的结构信息种类,同时尝试将TKGE-PN应用到除链路预测之外的其他知识图谱表示学习任务中。

本文涉及的研究内容还有许多可以拓展的地方。首先,知识图谱的图结构除了图路径和局部邻域之外还有其他的表现形式,我们计划进一步探索对于知识图谱中结构信息的利用,例如在规模更大的子图范围上来应用Transformer网络,进一步挖掘模型的潜力。

其次,除了结构信息之外,知识图谱中也蕴含着丰富的文本信息,例如实体的文字描述。而Transformer网络被广泛应用于预训练语言模型中,在挖掘文本语义信息方面存在天然优势。未来我们计划研究将模型与预训练语言模型进行结合,同时利用文本信息和图结构信息来完成知识图谱补全任务。

最后,Transformer网络的复杂度较高,训练和预测时的开销较大,在处理大规模知识图谱时可能会遇到资源上的瓶颈。因此如何优化基于Transformer的结构带来的大量资源开销也是一个值得研究的方向。



% 参考文献
% 手册中参考文献标准似乎并没有严格按照国标GBT7714-2015执行
\Bib{bst/GBT7714-BUAA}{ref}

% 攻读学位期间成果
% !TeX root = ../Template.tex
% [攻读学位期间取得的成果]
\achievement

\noindent
[1] \textbf{Zhu Tong}, Tan Huobin, Chen Xinyu, Ren Yating. A Transformer-Based Knowledge Graph Embedding Model Combining Graph Paths and Local Neighborhood[C]//2024 International Joint Conference on Neural Networks (IJCNN). Yokohama, Japan:Institute of Electrical and Electronics Engineers,2024.已录用 (CCF-C)

\noindent
[2] Liu Xiyang, \textbf{Zhu Tong}, Tan Huobin, Zhang Richong. Heterogeneous graph neural network with hypernetworks
for knowledge graph embedding [C]//The Semantic Web–ISWC 2022: 21st International
Semantic Web Conference, Virtual Event, October 23–27, 2022, Proceedings. Berlin,
Heidelberg: Springer-Verlag, 2022: 284–302 (CCF-B)

\noindent
[3] 郭子溢,\textbf{朱桐},林广艳,谭火彬. 球面坐标下基于语义分层的知识图谱补全方法 [J]. 应用科学学报, 2024, 42 (01): 119-133. (北大核心)



% 致谢
% !TeX root = ../Template.tex
% [致谢]
\acknowledgments

研究生三年生涯转瞬即逝,我的学生生涯也马上要走到了尽头。回首过去的三年,我初次叩开了科研的大门,虽然未能深入探索,但也收获良多,受益匪浅。

在三年的研究生生活中,我首先要感谢的是谭火彬老师和林广艳老师对我的学术和生活上给予的指导和帮助。两位老师不仅会在学术上为我提供建议,指明方向,对我的论文提出了许多宝贵的修改意见,生活中,还时刻关心着我们的身体健康和心理状况,在我的职业规划上也提供了大力的支持。实验室的良好氛围也为我提供了非常好的学习和研究的环境。能够接受两位老师的指导是我的幸运。还要感谢我的父母,他们的默默付出是我最坚强的后盾,父母的理解和爱护是我低谷时最重要的慰藉和支持。

此外,我还要感谢实验室的各位同门。感谢刘希阳学长在我初入实验室时提供的指导,帮助我迈出了学术研究的第一步,在我论文工作遇到困难时也对我提出了非常宝贵的意见和建议。还要感谢柳啸峰、郭子溢、朱伯同三位学长在我的学术研究中为我答疑解惑,学长们丰富的经验帮助我少走了不少弯路。感谢任博林和蒋沛宇两位学弟,论文实验的顺利完成离不开他们对于实验室服务器的维护。

最后,我要感谢我的小伙伴们,三年的时间我们一起度过了一段难忘的时光。




\vspace{5cm}

\end{document}
