\chapter{实验与验证}

为了验证本课题提出的NATLP算法和TKGE-PN算法的有效性,本文对两个方法分别进行了实验验证。本章首先对实验方案设计进行了介绍,包括实验采用的数据集、实验环境、评估策略;随后分别介绍了两个方法的对比算法,并对实验结果进行了详细的分析,包括整体性能分析、关键设计分析以及超参数影响分析。

\section{实验方案设计}

\subsection{实验数据集}
为了对NATLP模型和TKGE-PN模型进行评估,本文在两个标准基准数据集FB15k-237\upcite{FB15k-237}和WN18RR\upcite{ConvE}上进行了实验。FB15k-237是开源知识图谱Freebase\upcite{freebase}的子集,存储了有关电影、演员、奖项等等现实世界的常识信息的。WN18RR则是开源知识图谱WordNet\upcite{wordnet}的子集,包含了英文单词中的语义信息,例如同义、反义、单词概念的上下层等多种单词语义关系。为了避免测试集出现逆关系泄露的问题,两个数据集中所有的逆关系都已经被去除。两个数据集的统计数据如表\ref{dataset_statistics}所示,其中训练集用于模型参数训练,验证集用于模型超参数调优,测试集用于模型性能评估。值得注意的是,这两个数据集中的关系数量存在差异,FB15k-237包含了237种Freebase中的不同关系,而WN18RR的关系种类数量为11,总的来说,相比于FB15k-237,WN18RR数据集更加的稀疏。

\begin{table}[htbp]
  \renewcommand\arraystretch{1.5}
  \caption{数据集统计信息}
  \centering
  \begin{tabular}{*{3}{c}}
    \toprule
    数据集 & FB15k-237 & WN18RR\\
    \midrule
    \#实体数量  & 10541 & 40943 \\
    \#关系数量 & 237 & 11\\
    \#训练集数量 & 272115 &86835\\
    \#验证集数量 &17535 &3034\\
    \#测试集数量 &20466 &3134\\
    \#实体平均度数 &42.7 &4.5\\
    \bottomrule
  \end{tabular}
  \label{dataset_statistics}
\end{table}

\subsection{实验评估策略}

知识图谱的链路预测任务被定义为实体排序预测任务。测试集中的每个三元组将在两种不同的场景中进行链路预测评估:给定头实体和关系下的尾实体预测$(s,r,?)$,以及给定尾实体和关系下的头实体预测$(?,r,o)$。在实践中,头实体预测以$(o,r^-1,?)$的形式执行。预测时,待预测的头实体或者尾实体将被每个候选实体替换,并计算每个候选三元组的得分,随后所有的候选三元组将按照分数降序进行排序,以获得基本事实三元组的准确排名,并根据排名来对评估指标进行计算。

在知识图谱补全任务中,平均排名(Mean Rank,MR)、倒数平均排名(Mean Reciprocal Rank,MRR)和前N名百分比(Hits@n)三种评估指标将用来评估模型的性能。假设测试集中待预测三元组的数量为$K$,$rank_i$为第$i$个三元组的正确实体在所有候选实体中的排序位置,则平均排序MR的计算方式为:
\begin{equation}
    MR = \frac{\sum_i rank_i}{K}
\end{equation}
MR代表了所有正确实体的平均排序位置,MR越小,说明正确实体的得分越高,排名越靠前,模型的性能更好。但是MR存在的最大问题则是它对于不同排序位置的预测效果投入的关注是一样的,例如,假设有两个三元组,其中一个三元组的正确实体的排名从105上升到了100,另一个三元组的正确实体的排名从5上升到了1,他们对于MR指标的贡献是一致的,但是一般来说,在实验中我们认为从5到1的提升是更有意义的,平均排名指标MR则忽略了这一点。

平均倒数排名MRR对于这个问题进行了改进,MRR计算的时候正确实体排名的倒数的平均值,而不是排名的平均值,具体计算公式为:
\begin{equation}
    MRR = \frac{1}{K} \sum_i \frac{1}{rank_i}
\end{equation}
通过这样的计算方式,排名越靠前的项会获得更大的权重占比,对于整体性能指标的贡献越大。和MR不同,MRR越高,代表模型的性能越好。

前N名百分比Hits@n指的则是所有测试三元组中正确实体的排序处于前n名的比例,计算方式为:
\begin{equation}
    Hits@n = \frac{\sum_i \left[1 \ if \ rank_i \leq n \ or \ 0\right] }{K}
\end{equation}
指标越高,代表模型的性能越好。在知识图谱补全任务中,常用的前N名百分比指标包括Hits@1,Hits@3和Hits@10。

此外,注意到对于一个待预测的三元组$(s,r,?)$,正确实体$o$的选项可能不止一个,例如(姚明,出生于,上海)和(姚明,出生于,中国)都是正确的事实。在计算评价指标时,其余的正确实体可能会导致当前测试三元组中的正确实体排名下降,对模型评估造成影响,因此本文和大多数基线模型类似,计算排序时除了基本事实三元组之外的所有正确三元组都被排除在排名之外。

\subsection{实验环境}

实验中用到的服务器硬件和软件配置如表所示。硬件配置方面,实验所采用的服务器的处理器规格为